\section{Applications}
\label{sec:apps}
%
In this section, we will examine several related applications of \slog{}: implementing reduction systems, natural deduction systems, AAM-based program analyses, and natural-deduction-style type systems. 

We start with a $\lambda$-calculus interpreter. Let's observe how $\beta$-reduction can be defined via capture-avoiding substitution. If a \textbf{do-subst} fact is emitted where a reference to variable \texttt{x} is being substituted with expression \texttt{E}, associate it in the \textbf{subst} relation with \texttt{E}:
%
\begin{Verbatim}[baselinestretch=0.8,commandchars=\\\{\}]
(\rtag{subst} \huhclause{?(\rtag{do-subst} (\rtag{ref} x) x E)} E)
\end{Verbatim}
%
However, if \texttt{x} and \texttt{y} are distinct variables, the substitution yields expression \texttt{(ref x)} unchanged:
%
\begin{Verbatim}[baselinestretch=0.8,commandchars=\\\{\}]
[(=/= x y) --> (\rtag{subst} \huhclause{?(\rtag{do-subst} (\rtag{ref} x) y E)} (\rtag{ref} x))]
\end{Verbatim}
%
Recall that ?-clauses are body clauses, so these rules could also have been written more verbosely:
%
\begin{Verbatim}[baselinestretch=0.8,commandchars=\\\{\}]
[(= d (\rtag{do-subst} (\rtag{ref} x) x E)) --> (\rtag{subst} d E)]
[(=/= x y) (= d (\rtag{do-subst} (\rtag{ref} x) y E)) --> (\rtag{subst} d (\rtag{ref} x))]
\end{Verbatim}

At a lambda, where the formal parameter shadows the variable being substituted, its scope ends and substitution stops:
%
\begin{Verbatim}[baselinestretch=0.8,commandchars=\\\{\}]
(\rtag{subst} \huhclause{?(\rtag{do-subst} (\rtag{lam} x Ebody) x E)} (\rtag{lam} x Ebody))
\end{Verbatim}
%
If the variable does not match and is not free in the (argument) expression \texttt{E}, substitution may continue under the lambda, triggered by a ! clause:
%
\begin{Verbatim}[baselinestretch=0.8,commandchars=\\\{\}]
[(=/= x y)  ~(free E x)   
 --> (\rtag{subst} \huhclause{?(\rtag{do-subst} (\rtag{lam} x Ebody) y E)}
            (\rtag{lam} x \{\rtag{subst} \bangclause{!(\rtag{do-subst} Ebody y E)}\}))]
\end{Verbatim}
%
Three further syntactic extensions are being used in this rule. First off, the negated \texttt{\textasciitilde{}(free E x)} clause in the body requires that the compiler stratify computation of \texttt{free}, as normal when adding otherwise nonmonotonic rule dependance to Datalog programs.

Second, the process of rewriting the lambda body is triggered by the establishment of a \textbf{\texttt{do-subst}} fact via a ! clause. These ! clauses generate facts on-the-fly during rule evaluation, allowing other rules to hook-in by generating a fact to trigger them (using a ! clause) and expecting a response in the surrounding body clauses. These ! clauses are implemented by generating a subrule whose head clause is the intermediate ! clause and whose body contains all body clauses the ! clause depends upon, along with and any ? clauses in the rule (which are always required to trigger any rule). In this case, term-substitution rules respond to the \rtag{do-subst} request via the \textbf{\texttt{subst}} relation, as queried here by the \{ \!\} expression in \texttt{\{\rtag{subst} \bangclause{!(\rtag{do-subst} Ebody y E})}\}.

Third, this \{ \!\} syntax allows for looking up the final column of a relation by providing all but the final-column value. \texttt{(foo x \{bar y\})} desugars into \texttt{(and (foo x z) (bar y z))}, allowing for the looked-up value to be unified with the position of the \{ \!\} expression in a natural way. Do note that the relation need not actually be functional and could just as easily associate multiple values with any input.
%
\{ \!\} expressions and ! clauses are especially expressive when used together in this way for direct recursion.

If we were to desugar the \{ \!\} syntax, ? clause, and ! clause in this rule, we would obtain two rules.
%
The rule below on the left emits a \rtag{do-subst} fact for the body of the lambda, if it qualifies for rewriting, and a \rtag{ruleXX-midpoint} fact saving pertinent details of the rule needed in its second half. Below on the right, the second half of the rule requires that the first half of the rule triggered and that the \rtag{subst} relation has responded with a rewritten lambda body for the \rtag{do-subst} fact \texttt{do'}.
%
\begin{multicols}{2}
\begin{Verbatim}[baselinestretch=0.8,commandchars=\\\{\}]
[(=/= x y)  ~(free E x)
 (= do (\rtag{do-subst} (\rtag{lam} x Ebody) y E)) 
 -->
 (= do' (\rtag{do-subst} Ebody y E)) 
 (\rtag{ruleXX-midpoint} do do' x)]
\columnbreak
[(\rtag{ruleXX-midpoint} do do' x)
 (\rtag{subst} do' Ebody')
 --> 
 (\rtag{subst} do (\rtag{lam} x Ebody'))]]
\end{Verbatim}
\end{multicols}
%
Finally, in the case of an application, substitution is performed down both subexpressions.
%
If one ! clause were nested under the other, they would need to be ordered. In this case, the compiler will detect that both !-clause facts can be emitted in parallel, so this rule will also split into two rules as in the rule above. The first rule will generate both !-clause facts and the second rule will await a response for both.
%
In this way, \slog{}'s semantics for ! clauses assists to naturally enable exposure of parallelism in semantics it models.

\begin{wrapfigure}{l}{6.5cm}
\begin{Verbatim}[baselinestretch=0.8,commandchars=\\\{\}]
(\rtag{subst} \huhclause{?(\rtag{do-subst} (\rtag{app} Ef Ea) x E)}
       (\rtag{app} \{\rtag{subst} \bangclause{!(\rtag{do-subst} Ef x E)}\}
	    \{\rtag{subst} \bangclause{!(\rtag{do-subst} Ea x E)}\}))
\end{Verbatim}
\end{wrapfigure}
%
With a substitution function defined, we can define evaluation using a pair of relations: \rtag{interp} and \rtag{do-interp}. A lambda \texttt{(\rtag{lam} x body)} is already fully reduced. An application reduces its left-hand subexpression to a lambda, substitutes the argument for the formal parameter, and reduces the body. 

\begin{wrapfigure}{r}{6cm}
\vspace{-0.15cm}
\begin{Verbatim}[baselinestretch=0.8,commandchars=\\\{\}]
\comm{; values}
(\rtag{interp} \huhclause{?(\rtag{do-interp} (\rtag{lam} x body))}
        (\rtag{lam} x body))

\comm{; application}
[(\rtag{interp} \bangclause{!(\rtag{do-interp} fun)} (\rtag{lam} x body))
 (\rtag{subst} \bangclause{!(\rtag{do-subst} body x arg)} body')
 -->
 (\rtag{interp} \huhclause{?(\rtag{do-interp} (\rtag{app} fun arg))}
         \{\rtag{interp} \bangclause{!(\rtag{do-interp} body')}\})]
\end{Verbatim}
\end{wrapfigure}
%
The compiler will detect in this case that the \rtag{do-subst} !-clause fact depends on variable \texttt{body}, and the \rtag{do-interp} ! clause in the head depends on \texttt{body'}, but that the first \rtag{do-interp} !-clause fact only depends on the variable fun from the original ?-clause fact kicking off the rule. These three sequential ! clauses split the rule into four parts during compilation, just as a continuation-passing-style (CPS) tranformation~\cite{appel2007compiling} would explicitly break a traditional functional implementation of this recursive, substitution-based interpreter into one function entry point and three continuation entry points. Unlike traditional CPS translation of functional programs however, ! clauses in \slog{} will naturally emit multiple facts for parallel processing in a nonblocking manner when variable dependence allows for parallelism. 

\subsection{Abstract Machines}
\label{sec:apps:am}
%
Next, instead of using terms alone to represent intermediate points in evaluation, we may wish to explicitly represent facets of evaluation such as the environment, the stack, and the heap. Instead of representing environments through substitution, we may want to represent them explicitly in a higher-order way. As shown in Figure~\ref{fig:defun-env}, with first-class facts and ad hoc polymorphic rules, we can use defunctionalization to implement first-class relations, providing a global \rtag{env-map} relation, we can read with an \texttt{\{\rtag{env-map} env x\}} expression (assuming ground variables \texttt{env} and \texttt{x}), along with a \texttt{(ext-env env x val)} facility for deriving an extended environment.
%
\begin{figure*}[h]
\begin{Verbatim}[baselinestretch=0.8,commandchars=\\\{\}]
\comm{; environments (defunctionalized)}
(\rtag{env-map} \huhclause{?(\rtag{ext-env} env x val)} x val)
[(=/= x y) --> (\rtag{env-map} \huhclause{?(\rtag{ext-env} env x _)} y \{\rtag{env-map} env y\})]
\end{Verbatim}
\caption{Defunctionalized environments; extension via \texttt{(ext-env env x v)}, lookup via \texttt{\{env-map env x\}}.}
\label{fig:defun-env}
\end{figure*}


On the left of Figure~\ref{fig:ce-machines} shows an abstract machine for CBN evaluation, and on the right, an abstract machine for CBV evaluation.
%
At the top, the rules for reference use \texttt{\{\rtag{env-map} env x\}} to access the value from the defunctionalized
\rtag{env-map} relation. In the CBN version, we cannot count on the stored closure to be a lambda closure, so we continue
interpretation, using another \{ \!\} expression to drop-in the transitive reduction of the stored argument closure.
%
Lambda closures are the base case which \rtag{interp} as themselves.
%
Finally, application closures trigger a closure to evaluate \texttt{Ef} via a ! clause, \texttt{\bangclause{!(clo Ef env)}}, and the lambda closure that finally results has its body evaluated under its environment, extended with parameter mapped to argument.
%
In the CBN interpreter, \texttt{(ext-env env' x (clo Ea env))} puts the argument expression \texttt{Ea} in the environment, closed with the current environment. In the CBV interpreter, \texttt{(ext-env env' x Eav)} puts the argument value \texttt{Eav} in the environment (after first evaluating it). In both these interpreters, the \rtag{app}-handling rules use ! clauses to implicitly create handling rules and a chain of continuation facts so \rtag{interp} maybe be utilized in a direct-recursive manner. The ! syntax introduces a CPS-like transformation that provides a stack in the interpretation of \slog{} rules for these CE interpreters to map their stack onto.
%
\begin{figure*}[h]
\begin{multicols}{2}
\vspace{-0.3cm}
\begin{Verbatim}[baselinestretch=.75,commandchars=\\\{\}]
\comm{; ref}
(\rtag{interp} \huhclause{?(\rtag{clo} (\rtag{ref} x) env)}
        \{\rtag{interp} \{\rtag{env-map} env x\}\})
\comm{; lam}
(\rtag{interp} \huhclause{?(\rtag{clo} (\rtag{lam} x Eb) env)}
        (\rtag{clo} (\rtag{lam} x Eb) env))
\comm{; app}
[(\rtag{interp} \bangclause{!(\rtag{clo} Ef env)}
         (\rtag{clo} (\rtag{lam} x Eb) env'))
 (= env'' (\rtag{ext-env} env' x (\rtag{clo} Ea env)))       
 (\rtag{interp} \bangclause{!(\rtag{clo} Eb env'')} v)
 -->
 (\rtag{interp} \huhclause{?(\rtag{clo} (\rtag{app} Ef Ea) env)} v)]  
\columnbreak
\comm{; ref }
(\rtag{interp} \huhclause{?(\rtag{clo} (\rtag{ref} x) env)}
        \{\rtag{env-map} env x\})
\comm{; lam}
(\rtag{interp} \huhclause{?(\rtag{clo} (\rtag{lam} x Eb) env)}
        (\rtag{clo} (\rtag{lam} x Eb) env))
\comm{; app}
[(\rtag{interp} \bangclause{!(\rtag{clo} Ef env)}
         (\rtag{clo} (\rtag{lam} x Eb) env'))
 (\rtag{interp} \bangclause{!(\rtag{clo} Ea env)} Eav)
 (\rtag{interp} \bangclause{!(\rtag{clo} Eb (\rtag{ext-env} env' x Eav))} v)
 -->
 (\rtag{interp} \huhclause{?(\rtag{clo} (\rtag{app} Ef Ea) env)} v)]
\end{Verbatim}
\end{multicols}
\caption{Two CE (closure-creating) interpreters in \slog{}; for CBN eval. (left) and CBV eval. (right).}
\label{fig:ce-machines}
\end{figure*}


\begin{figure*}[h]
\vspace{-0.75cm}
\hspace{-0.95cm}  
\begin{minipage}{0.98\linewidth}
\begin{multicols}{2}
\begin{Verbatim}[baselinestretch=.75,commandchars=\\\{\}]
    
\comm{; eval ref}
(\rtag{interp} \huhclause{?(\rtag{cek} (\rtag{clo} (\rtag{ref} x) env) k)}
        \{\rtag{interp} \bangclause{!(\rtag{cek} {\rtag{env-map} env x} k)}\})

      
\comm{; eval lam (apply)}
(\rtag{interp} \huhclause{?(\rtag{cek} (\rtag{clo} (\rtag{lam} x Eb) env)}
              \huhclause{[aclo k ...])}
        \{\rtag{interp} \bangclause{!(\rtag{cek} (\rtag{clo} Eb}
                       \bangclause{(\rtag{ext-env} env x aclo))}
                  \bangclause{k)}\})

                  

                  
\comm{; eval app}
(\rtag{interp} \huhclause{?(\rtag{cek} (\rtag{clo} (\rtag{app} Ef Ea) env) k)}
        \{\rtag{interp} \bangclause{!(\rtag{cek} (\rtag{clo} Ef env)}
                      \bangclause{[(\rtag{clo} Ea env) k ...])}\})
\comm{; return / halt}
(\rtag{interp} \huhclause{?(\rtag{cek} (\rtag{clo} (\rtag{lam} x Eb) env) [])}
        (\rtag{clo} (\rtag{lam} x Eb) env))

\columnbreak
\comm{; eval ref}
(\rtag{interp} \huhclause{?(\rtag{cek} (\rtag{clo} (\rtag{ref} x) env) k)}
        \{\rtag{interp} \bangclause{!(\rtag{cek} {\rtag{env-map} env x} k)}\})
\comm{; eval lam (ret to ar-k)}
(\rtag{interp} \huhclause{?(\rtag{cek} (\rtag{clo} (\rtag{lam} x Eb) env)}
              \huhclause{(\rtag{ar-k} aclo k))}
        \{\rtag{interp} \bangclause{!(\rtag{cek} aclo}
                      \bangclause{(\rtag{fn-k} (\rtag{clo} (\rtag{lam} x Eb) env)}
                            \bangclause{k))}\})
\comm{; eval lam (ret to fn-k)}
(\rtag{interp} \huhclause{?(\rtag{cek} (= aclo (\rtag{clo} (\rtag{lam} _ _) _))}
              \huhclause{(\rtag{fn-k} (\rtag{clo} (\rtag{lam} x Eb) env) k))}
        \{\rtag{interp} \bangclause{!(\rtag{cek} (\rtag{clo} Eb}
                           \bangclause{(\rtag{ext-env} env x aclo))}
                      \bangclause{k)}\})
\comm{; eval app}
(\rtag{interp} \huhclause{?(\rtag{cek} (\rtag{clo} (\rtag{app} Ef Ea) env) k)}
        \{\rtag{interp} \bangclause{!(\rtag{cek} (\rtag{clo} Ef env)}
                      \bangclause{(\rtag{ar-k} (\rtag{clo} Ea env) k))}\})
\comm{; return to (halt-k)}
(\rtag{interp} \huhclause{?(\rtag{cek} (\rtag{clo} (\rtag{lam} x Eb) env) (\rtag{halt-k}))}
        (\rtag{clo} (\rtag{lam} x Eb) env))
\end{Verbatim}
\end{multicols}
\end{minipage}
\caption{Two CEK (stack-passing) interpreters in \slog{}; for CBN eval. (left) and CBV eval. (right).}
\label{fig:cek-machines}
\end{figure*}

%
We can also implement the stack ourselves within our interpreter, thereby eliminating its need for our interpreter itself, by applying a stack-passing transformation.
On the left, Figure~\ref{fig:cek-machines} shows Krivine's machine~\cite{krivine:2007:cbn}, a tail-recursive abstract machine for CBN evaluation, and on the right, a tail-recursive abstract machine for CBV evaluation. Each of these machines incrementally constructs and passes a stack. In the CBN stack-passing interpreter, each application reached pushes a closure for the argument expression onto the stack. When a lambda is reached, this continuation is handled by popping its latest closure, the argument value. In the CBV stack-passing interpreter, each application reached pushes an \rtag{ar-k} continuation frame on the stack to save the argument value and environment. Whan a lambda is reached, this continuation is handled by swapping it for a \rtag{fn-k} continuation that saves the function value while the argument expression is evaluated (before application). Finally, when a lambda is reached, the \rtag{fn-k} continuation is handled by applying the saved closure. Now that the stack is entirely maintained by the interpreter itself, you may note that all recursive uses of \texttt{\{interp (cek ...)\}} are in tail position for the result column of relation \rtag{interp}. 







\subsection{Abstracting Abstract Machines}
\label{sec:apps:aam}
%
\begin{wrapfigure}{r}{6.8cm}
\vspace{-0.75cm}
\begin{Verbatim}[baselinestretch=.75,commandchars=\\\{\}]
\comm{; eval ref -> ret}
[(\rtag{eval} (\rtag{ref} x) env sto k c)
 --> 
 (\rtag{ret} \{\rtag{sto-map} sto \{\rtag{env-map} env x\}\} sto k c)] 
\comm{; eval lam -> ret}
[(\rtag{eval} (\rtag{lam} x Eb) env sto k c)
 --> 
 (\rtag{ret} (\rtag{clo} (\rtag{lam} x Eb) env) sto k c)]
\comm{; eval app -> eval}
[(\rtag{eval} (\rtag{app} Ef Ea) env sto k c)
 --> 
 (\rtag{eval} Ef env sto (\rtag{ar-k} Ea env k) c)]
\comm{; ret to kaddr -> ret}
[(\rtag{ret} vf sto (\rtag{kaddr} c') c)
 -->
 (\rtag{ret} vf sto \{\rtag{sto-map} sto (\rtag{kaddr} c')\} c)]
\comm{; ret to ar-k -> eval}
[(\rtag{ret} vf sto (\rtag{ar-k} Ea env k) c)
 --> 
 (\rtag{eval} Ea env sto (\rtag{fn-k} vf k) c)]
\comm{; ret to fn-k -> apply}
[(\rtag{ret} va sto (\rtag{fn-k} vf k) c)
 --> 
 (\rtag{apply} vf va sto k c)]
\comm{; apply -> eval}
[(\rtag{apply} (\rtag{clo} (\rtag{lam} x Eb) env) va sto k c)
 --> 
 (\rtag{eval} Eb
       (\rtag{ext-env} env x (\rtag{addr} c))
       (\rtag{ext-sto} (\rtag{ext-sto} sto (\rtag{kaddr} c) k)
                (\rtag{addr} c) va)
       k
       \{+ 1 c\})]
\end{Verbatim}
\caption{A CESKT (control, environment, store, kontinuation, timestamp) interpreter in \slog{}.}
\label{fig:cesk-machine}
\vspace{-0.25cm}
\end{wrapfigure}

%
The \emph{abstracting abstract machines} (AAM) methodology \cite{might2010abstract,VanHorn:2010} proscribes a particular systematic application of abstract interpretation \cite{cousot77unifiedmodel,cousot1996abstract,cousot1979systematic} on abstract-machine operational semantics like those we've just built in \slog{}.
%
AAM proposes key preparatory refactorings of an abstract machine, to remove direct sources of unboundedness through recursion, before more straightforward structural abstraction can be applied.
%
In particular, there are two main sources of unboundedness in the CEK machines: environments and continuations. Environments contain closures which themselves contain environments; continuations are a stack of closures formed inductively in the CBV CEK machine and formed using \slog{}'s list syntax in the CBN CEK machine to more closely follow the usual presentation of Krivine's machine~\cite{krivine:2007:cbn}. 
%
AAM proposes threading each such fundamental source of unboundedness through a store, added in a normal store-passing transformation of the interpreter that might be used to add direct mutation or other effects to the language. Environments will map variables to addresses in the store, not to closures directly, and the stack will be store allocated at least once per function application so the stack may not grow indefinitely without the store likewise growing without bound. These two changes will permit us to place a bound on the addresses allocated, and thereby finitize the machine's state space as a whole.


Figure~\ref{fig:cesk-machine} shows the CBV CEK machine of Figure~\ref{fig:cek-machines} modified in a few key ways, yielding a CESKT machine with control expression, environment, store, continuation, and timestamp/contour components:
%
\begin{itemize}
\item
  \emph{abstract-machine states have been factored} into \rtag{eval}, \rtag{apply}, and \rtag{ret} configurations; an \rtag{eval} state has a control expression, environment (mapping variables to addresses), store (mapping addresses to closures and continuations), current continuation, and timestamp (tracking the size of the store, and thus the next address); an \texttt{apply} state has a closure being applied, argument value, store, continuation, and timestamp; and a \texttt{ret} state has a value being returned, a store, a continuation, and a timestamp;  
\item
  \emph{state transitions have been written as small-step rules} that always terminate; previously, our CEK machines were written to take a big-step from \rtag{cek}-state to the final, denoted value as logged in the \texttt{(\rtag{interp} e v)} relation, but in tail-recursive fashion, using ! clauses); Figure~\ref{fig:cesk-machine} has no explicit small-step relation, but simply says, for example, that the existence of a \rtag{ret} state permits us to deduce to existence of an \rtag{apply} state; if we were to want an explicit \rtag{step} relation, we could again give this rule a presentation with an implied body via a ? clause; for example:
\begin{Verbatim}[baselinestretch=.75,commandchars=\\\{\}]
(\rtag{step} \huhclause{?(\rtag{ret} va sto (\rtag{fn-k} vf k) c)}
      (\rtag{apply} vf va sto k c))
\end{Verbatim}
\item
  \emph{states have been subjected to a store-passing transformation} which has added a store \texttt{sto} and timestamp (stored-value count) \texttt{c} to each state; environments now bind variables to addresses and the current store binds those addresses to values; we perform a variable lookup with \texttt{\{\rtag{sto-map} sto \{\rtag{env-map} env x\}\}}; at an \rtag{apply} state, we use the store count \texttt{c} to generate a fresh address \texttt{(\rtag{addr} c)} for the parameter \texttt{x}; we also store-allocate the current continuation at a continuation address \texttt{(\rtag{kaddr} c)}, in preparation for modeling the stack finitely as well; when returning to a \texttt{(\rtag{kaddr} c)}, the continuation is simply fetched from the store as in the fourth rule down (ret to kaddr).
\end{itemize}

From here it suffices to pick a finite set from which to draw addresses. To instantiate a monovariant control-flow analysis from this CESKT interpreter, it would be enough to use the variable name itself as the address or to generate an address \texttt{(\rtag{addr} x)}. When the environment and store become finite, so does the number of possible states. Consider what happens, as the naturally relational \rtag{sto-map} relation encoding stores conflates multiple values at a single address for the same variable. Conflation in the store would lead naturally to nondeterminism in any \rtag{step} relation. When looking up a variable, two distinct \rtag{ret} states could result, leading to two distinct \rtag{apply} states after some further steps.

A (potentially) more precise, though (potentially) more costly analysis would be to specialize all control-flow points and store points by a finite history of recent or enclosing calls. Such a \emph{$k$-call-sensitive} analysis can be instantiated using a specific instrumentation and allocation policy, as can many others~\cite{gilray2016poly}. It requires an instrumentation to track a history of $k$ enclosing calls, and then an \emph{abstract allocation policy} that specializes variables by this call history at binding time. Such context-sensitive techniques are a gambit that the distinction drawn between variable \texttt{x} when bound at one call-site vs another will prove meaningful---in that it may correlate with its distinct values. Increasing the polyvariance allows for greater precision while also increasing the upper-bound on analysis cost. In a well known paradox of programming analyses, greater precision sometimes goes hand-in-hand with lower cost in practice because values that are simpler and fewer are simpler to represent~\cite{wright1998polymorphic}. At the same time, we use the polyvariant entry point of each function, its body and abstract contour---\texttt{(\rtag{kaddr} Eb c')}---to store allocate continuations as suggested by previous literature on selecting this address~\cite{gilray2016p4f} so as to adapt to the value polyvariance chosen.

The per-state store by itself is a source of exponential blowup for any polyvariant control-flow analysis~\cite{midtgaard2012control}. Instead, it is standard to use a global store and compute the least-upper-bound of all per-state stores. In \slog{} this is as simple as using a single global \texttt{(\rtag{store} addr val)} relation instead of a defunictionalized \texttt{(\rtag{sto-map} sto addr val)} relation that approximates all per-state stores in one. The left side of Figure~\ref{fig:kcfa-mcfa} shows a version of the CESKT machine with a global store and a tunable instrumentation that can be varied by changing the \rtag{tick} function rule. Currently, \rtag{tick} instantiates this to a $3$-$k$-CFA: at each function application, the current call site (now saved in the \rtag{ar-k} and \rtag{fn-k} continuation frames to provide to the apply state) is saved in front of the current call history and the fourth-oldest call is dropped.

This is the classic $k$-CFA, except perhaps that the original $k$-CFA, formulated for CPS as it was, also tracked returns positively instead of reverting the timestamp as functions return like we do here. The original $k$-CFA used true higher-order environments, unlike equivalent analyses written for object oriented languages which implicitly had flat environments (objects)~\cite{might2010resolving}. The corresponding CFA for functional languages is called $m$-CFA and is shown on the right side of Figure~\ref{fig:kcfa-mcfa}. $m$-CFA has only the latest call history as a flat context. Instead of having a per-variable address with a per-variable history tracked by a per-state environment, $m$-CFA stores a variable \texttt{x} at abstract contour \texttt{c} (i.e., abstract timestamp, instrumentation, 3-limited call-history) in the store at the address \texttt{(\rtag{addr} x c)}. This means at every update to the current flat context \texttt{c}, now taking the place of the environment, all free variables must be propagated into an address \texttt{(\rtag{addr} x c)}.





\begin{figure*}
%\vspace{-0.5cm}
\begin{multicols}{2}
\begin{Verbatim}[baselinestretch=.75,commandchars=\\\{\}]
\comm{;; Eval states}
[(\rtag{eval} (\rtag{ref} x) env k _)
 -->
 (\rtag{ret} \{\rtag{store} \{\rtag{env-map} env x\}\} k)]
[(\rtag{eval} (\rtag{lam} x body) env k _)
 -->
 (\rtag{ret} (\rtag{clo} (\rtag{lam} x body) env) k)]
[(\rtag{eval} (\rtag{app} ef ea) env k c)
 -->
 (\rtag{eval} ef env
         (\rtag{ar-k} ea env (\rtag{app} ef ea) c k)
         c)]
\comm{;; Ret states}
[(\rtag{ret} vf (\rtag{ar-k} ea env call c k))
 -->
 (\rtag{eval} ea env (\rtag{fn-k} vf call c k) c)]
[(\rtag{ret} va (\rtag{fn-k} vf call c k))
 -->
 (\rtag{apply} call vf va k c)]
[(\rtag{ret} v (\rtag{kaddr} e env))
 (\rtag{store} (\rtag{kaddr} e env) k)
  --> 
 (\rtag{ret} v k)]
\comm{;; Apply states}
[(\rtag{apply} call (\rtag{clo} (\rtag{lam} x Eb) env) va k c)
 -->
 (\rtag{eval} Eb env' (\rtag{kaddr} Eb env') c')
 (\rtag{store} (\rtag{kaddr} Eb env') k)
 (\rtag{store} (\rtag{addr} x c') va)
 (= env' (\rtag{ext-env} env x (\rtag{addr} x c')))
 (= c' \{\rtag{tick} \bangclause{!(\rtag{do-tick} call c)}\})]

\comm{;; tick (tuning for 3-k-CFA)}
(\rtag{tick} \huhclause{?(\rtag{do-tick} call [h0 h1 _])}
      [call h0 h1])

\columnbreak
\comm{;; Eval states}
[(\rtag{eval} (\rtag{ref} x) k c)
 -->
 (\rtag{ret} \{\rtag{store} (\rtag{addr} x c)\} k)]
[(\rtag{eval} (\rtag{lam} x body) k c)
 -->
 (\rtag{ret} (\rtag{clo} (\rtag{lam} x body) c) k)]
[(\rtag{eval} (\rtag{app} ef ea) k c)
 -->
 (\rtag{eval} ef (\rtag{ar-k} ea (\rtag{app} ef ea) c k) c)]


\comm{;; Ret states}
[(\rtag{ret} vf (\rtag{ar-k} ea call c k))
 -->
 (\rtag{eval} ea (\rtag{fn-k} vf call c k) c)]
[(\rtag{ret} va (\rtag{fn-k} vf call c k))
 -->
 (\rtag{apply} call vf va k c)]
[(\rtag{ret} v (\rtag{kaddr} e c))
 (\rtag{store} (\rtag{kaddr} e c) k)
 -->
 (\rtag{ret} v k)]
\comm{;; Apply states}
[(\rtag{apply} call (\rtag{clo} (\rtag{lam} x Eb) _) va k c)
 -->
 (\rtag{eval} Eb (\rtag{kaddr} Eb c') c')
 (\rtag{store} (\rtag{kaddr} Eb c') k)
 (\rtag{store} (\rtag{addr} x c') va)
 (= c' \{\rtag{tick} \bangclause{!(\rtag{do-tick} call c)}\})]
\comm{; Propagate free vars}
[(\rtag{free} y (lam x body))
 (\rtag{apply} call (\rtag{clo} (\rtag{lam} x body) clam) _ _ c)
 -->
 (\rtag{store} (\rtag{addr} y \{\rtag{tick} \bangclause{!(\rtag{do-tick} call c)}\})
        \{\rtag{store} (\rtag{addr} y clam)\})]
\end{Verbatim}
\end{multicols}
\caption{An AAM for global-store $k$-CFA (left) and $m$-CFA (right) in \slog{}. These are evaluated in Section~\ref{sec:eval}.}
\label{fig:kcfa-mcfa}
\end{figure*}








\subsection{Type Systems}
\label{sec:apps:ts}

Along with operational semantics and program analyses, \slog{}
naturally enables the realization of structural type systems based on
constructive logics~\cite{tapl,atapl}.  These systems are often
specified via an inductively-defined typing judgment, whose
derivations may be represented in \slog{} via (sub)facts and whose
typing rules may be realized as rules in \slog{} (providing their
evaluation may be operationalized via \slog{}'s idioms). For example,
the rules for the simply-typed $\lambda$-calculus (STLC) in
Figure~\ref{fig:stlc} define the judgment $\Gamma \vdash e
:\tau$---under typing environment $\Gamma$, $e$ has been proven to
have type $\tau$. Proofs of this judgment are represented via \slog{}
facts of the form \texttt{(\rtag{:} (\rtag{ck} $\Gamma$ e) T)}; each
rule in the type system is then mirrored by a corresponding rule in
\slog{}.

When implementing a type system in \slog{}, it is crucial to consider
some important differences between \slog{} and natural deduction
per-se. First, equivalence in \slog{} is intensional, via fact
interning (as in type theories such as Coq's Calculus of Inductive
Constructions). For example, while our type checker for STLC decides
$\Gamma \vdash e :\tau$ via structural recursion on $e$, there are
infinitely many $\Gamma' \supseteq \Gamma$ for which $\Gamma' \vdash e
:\tau$ also holds---materializing these (infinite) $\Gamma'$
would result in nontermination.

Here, we focus on the presentation of algorithmic (i.e.,
syntax-directed) type checking procedures. The decidability of these
systems follows immediately from their structurally-recursive nature,
a property inherited by their \slog{} counterparts. We expect
enumerating terms in other theories which enjoy strong normalization
will readily follow. We anticipate \slog{}'s declarative style may
also be a natural fit for type synthesis, by bounding (potentially
infinite) rewritings using a decreasing ``fuel'' parameter. However,
we leave this, along with explorations of other (bidirectional,
substructrural, etc...)  type systems in \slog{} to future work.

\paragraph*{Simply-typed $\lambda$-calculus}

\begin{figure}
\begin{displaymath}
\begin{tabular}{lrcllrcl}
\textit{STLC Terms}& $e$ & $::=$ & $\big(\lambda (x\!:\!\tau)\,e\big)$ & \textit{STLC Types}& $\tau \in T$ & $::=$& $\tau \rightarrow \tau$ \\
&     &  $\mid$ & $(e_0 ~ e_1)$ & && $\mid$& \textsf{nat} \\
&     &  $\mid$ & $x$ & & & $\mid$ & $...$ \\
\end{tabular}
\end{displaymath}

\begin{flushleft}
\fbox{$\Gamma \vdash e : \tau$}
\end{flushleft}
\begin{tabular}{c|c}
%\toprule \\
 {{\small\textsc{T-Var}}\quad\quad{\LARGE ${ \frac{x : T \, \in\, \Gamma}{\Gamma\, \vdash\, x : T}}$}} \quad&
\begin{minipage}{2.5in}
\begin{Verbatim}[baselinestretch=.8,commandchars=\\\{\},codes={\catcode`$=3\catcode`^=7}]
[-->;-------- T-Var \\
 (\rtag{:} \huhclause{?(\rtag{ck} $\Gamma$ (\rtag{ref} x))} \{\rtag{env-map} $\Gamma$ x\})]
\end{Verbatim}
\end{minipage}
\\
\\
{{\small\textsc{T-Abs}}\quad\quad{\LARGE ${ \frac{\Gamma, x\, : \,T_1 \, \vdash \, e \, : \, T_2}{(\lambda\,(x\,:\, T_1)\,e)\, : T_1 \rightarrow \,T_2}}$}} \quad& 
\begin{minipage}{2.5in}
\begin{Verbatim}[baselinestretch=.8,commandchars=\\\{\},codes={\catcode`$=3\catcode`^=7}]
[(\rtag{:} \bangclause{!(\rtag{ck} (\rtag{ext-env} $\Gamma$ x T1) e)} T2)
 -->;-------- T-Abs
 (\rtag{:} \huhclause{?(\rtag{ck} $\Gamma$ (\rtag{$\lambda$} x T1 e))} (\rtag{->} T1 T2))]
\end{Verbatim}
\end{minipage}
\\
\\
 
{{\small\textsc{T-App}}\quad\quad{\LARGE ${ \frac{{\Gamma \, \vdash \, e_0\,:\,T_0\, \rightarrow\, T_1}\quad e_1\,:\,T_0}{\Gamma \, \vdash \, (e_0~e_1)\,:\,T_1}}$ }}\quad& 
\begin{minipage}{2.5in}
\begin{Verbatim}[baselinestretch=.8,commandchars=\\\{\},codes={\catcode`$=3\catcode`^=7}]
[(\rtag{:} \bangclause{!(ck $\Gamma$ e0)} (\rtag{->} T0 T1))
 (\rtag{:} \bangclause{!(ck $\Gamma$ e1)} T0)
 -->;-------- T-App
 (\rtag{:} \huhclause{?(\rtag{ck} $\Gamma$ (\rtag{app} e0 e1))} T1)]
\end{Verbatim}
\end{minipage} \\
%\bottomrule
\end{tabular}

\caption{Syntax (top) and semantics (left) of STLC; equivalent \slog{} (right).}
\label{fig:stlc}
\end{figure}

%% \begin{figure}
%% \begin{displaymath}
%% \begin{tabular}{lrcllrcl}
%% \textit{STLC Terms}& $e$ & $::=$ & $\big(\lambda (x\!:\!\tau)\,e\big)$ & \textit{STLC Types}& $\tau \in T$ & $::=$& $\tau \rightarrow \tau$ \\
%% &     &  $\mid$ & $(e_0 ~ e_1)$ & && $\mid$& $...$ \\
%% &     &  $\mid$ & $x$ & & &  \\
%% \textit{\lf{} Contexts}& $\Gamma$ & $::=$ & $\varnothing$ &\textit{\lf{} Types} & $\tau \in T$ & $::=$& $X$ \\
%% &     &  $\mid$ & $\Gamma, x \! : \! T$ & && $\mid$& $\Pi x \! : \! T.~ T$ \\
%% &     &  $\mid$ & $\Gamma, x \! ::\!  K$ & &&$\mid$& $(T~ e)$ \\
%% \textit{\lf{} Kinds} & $K$ & $::=$ & $\Pi x \! : \! T. K \mid \ast$ \\ 
%% \end{tabular}
%% \end{displaymath}
%% \caption{Syntax of STLC and \lf{}.}
%% \label{fig:syntax}
%% \end{figure}


%% \begin{figure}
%% \begin{flushleft}
%% \fbox{$\Gamma \vdash e : \tau$}
%% \end{flushleft}
%% \begin{tabular}{c|c}
%% %\toprule \\
%%  {{\small\textsc{T-Var}}\quad\quad{\LARGE ${ \frac{x : T \, \in\, \Gamma}{\Gamma\, \vdash\, x : T}}$}} \quad&
%% \begin{minipage}{2.5in}
%% \begin{Verbatim}[baselinestretch=.8,commandchars=\\\{\},codes={\catcode`$=3\catcode`^=7}]
%% [-->;-------- T-Var \\
%%  (\rtag{:} \huhclause{?(\rtag{ck} $\Gamma$ (\rtag{ref} x))} \{\rtag{env-map} $\Gamma$ x\})]
%% \end{Verbatim}
%% \end{minipage}
%% \\
%% \\
%% {{\small\textsc{T-Abs}}\quad\quad{\LARGE ${ \frac{\Gamma, x\, : \,T_1 \, \vdash \, e \, : \, T_2}{(\lambda\,(x\,:\, T_1)\,e)\, : T_1 \rightarrow \,T_2}}$}} \quad& 
%% \begin{minipage}{2.5in}
%% \begin{Verbatim}[baselinestretch=.8,commandchars=\\\{\},codes={\catcode`$=3\catcode`^=7}]
%% [(\rtag{:} \bangclause{!(\rtag{ck} (\rtag{ext-env} $\Gamma$ x T1) e)} T2)
%%  -->;-------- T-Abs
%%  (\rtag{:} \huhclause{?(\rtag{ck} $\Gamma$ (\rtag{$\lambda$} x T1 e))} (\rtag{->} T1 T2))]
%% \end{Verbatim}
%% \end{minipage}
%% \\
%% \\
 
%% {{\small\textsc{T-App}}\quad\quad{\LARGE ${ \frac{{\Gamma \, \vdash \, e_0\,:\,T_0\, \rightarrow\, T_1}\quad e_1\,:\,T_0}{\Gamma \, \vdash \, (e_0~e_1)\,:\,T_1}}$ }}\quad& 
%% \begin{minipage}{2.5in}
%% \begin{Verbatim}[baselinestretch=.8,commandchars=\\\{\},codes={\catcode`$=3\catcode`^=7}]
%% [(\rtag{:} \bangclause{!(ck $\Gamma$ e0)} (\rtag{->} T0 T1))
%%  (\rtag{:} \bangclause{!(ck $\Gamma$ e1)} T0)
%%  -->;-------- T-App
%%  (\rtag{:} \huhclause{?(\rtag{ck} $\Gamma$ (\rtag{app} e0 e1))} T1)]
%% \end{Verbatim}
%% \end{minipage} \\
%% %\bottomrule
%% \end{tabular}
%% \caption{Simply-Typed Lambda Calculus (TAPL Fig. 9.1) and \slog{} Equivalent}
%% \label{fig:stlc}
%% \end{figure}


We begin with the simply-typed
$\lambda$-calculus~\cite{barendregt2013lambda,tapl}.  The syntax of
STLC terms and types is shown at the top of Figure~\ref{fig:stlc}. Our
presentation roughly follows Chapter~9 of Benjamin Pierce's
\textit{Types and Programming Languages}~\cite{tapl}; we use
Scheme-style syntax to more closely mirror the \slog{}
presentation. STLC extends the untyped $\lambda$-calculus:
$\lambda$-abstractions are annotated with types, and variable typing
defers to a typing environment which assigns types to type variables
and is subsequently extended at callsites.  STLC defines a notion of
``simple'' types including (always) arrow types between simple types
and (sometimes) base types (e.g., \textsf{nat}), depending on the
presentation.
 
The key creative challenge in translating a type system into \slog{}
lies in the operationalization of its control flow. For example,
consider the \textsc{T-Var} rule on the left side of
Figure~\ref{fig:stlc}: the rule is parametric over the typing
environment $\Gamma$, however a terminating analysis necessarily
inspects only a finite number of typing environments. The solution is
interpret the rules in a demand-driven way, using subfacts to defer
type checking until necessary. Operationally, each type rule is
predicated upon a message \texttt{\huhclause{?(\rtag{ck} $\Gamma$
    e)}}, which triggers type synthesis for \texttt{e} under the
typing environment $\Gamma$.  Using this technique, we may invoke the
analysis by including a fact \texttt{(\rtag{ck} $\Gamma$ e)}. Using
stratified negation, we may treat \texttt{(\rtag{:} (\rtag{ck}
  $\Gamma$ e) $\tau$)} as a decision procedure, like so:

\begin{Verbatim}[baselinestretch=1,commandchars=\\\{\},codes={\catcode`$=3\catcode`^=7}]
[(\rtag{success} $\Gamma$ e $\tau$) <-- (\rtag{typecheck} $\Gamma$ e $\tau$) (\rtag{:} \bangclause{!(\rtag{ck} $\Gamma$ e)} $\tau$)]
[(\rtag{failure} $\Gamma$ e $\tau$) <-- ~(\rtag{success} $\Gamma$ e $\tau$)]
\end{Verbatim}

Here, the inclusion of \texttt{(\rtag{typecheck} $\Gamma$ e $\tau$)}
forces the materialization of \texttt{\bangclause{!(\rtag{ck} $\Gamma$
    e)}}, and subsequently the enumeration of the type of each of its
subexpressions. Once the resulting type is materialized, we force its
unification with $\tau$ (implicitly using \slog{}'s intensional notion
of equality) and, when successful, generate \texttt{(\rtag{success}
  $\Gamma$ e $\tau$)}. The resulting \slog{} implementation
necessarily terminates because the analysis enumerates at most a
finite number of \texttt{(\rtag{ck} $\Gamma$ e)} facts (because of the
structural recursion on $e$ done by \rtag{:}), which in-turn forces
materialization of a finite number of \texttt{(\rtag{ext-env}
  $\Gamma$ x T)} facts, each of which forces a bounded
materialization of \rtag{env-map}.

\paragraph*{Natural Deduction and Per Martin-L\"of's Type Theory}

The well-known Curry-Howard isomorphism relates terms in pure
functional languages to proofs in an appropriate constructive
logic~\cite{Curry:1934}. For STLC, the Curry-Howard isomorphism tells
us that we may read our type checker as a decision procedure for
intuitionistic propositional logic. As an alternative (but equivalent)
perspective, we now consider how \slog{} may represent proofs in Per
Martin-L\"of's intuitionistic type theory (ITT)~\cite{Martin-Lof1996}.

By design, ITT cleanly separates propositions from their associated
derivations (proof objects). In \slog{}, we may represent derivations
as structured facts, obtaining the nested structure of derivations via
\slog{}'s subfacts. Adopting this perspective, checking a
natural-deduction proof in ITT involves propagating assumptions to
their usages (akin to the propagation achieved using maps in STLC).

\begin{figure}[h!]

\begin{tabular}{c|c}
{{\textsc{$\land$I}}\quad{\LARGE  ${ \frac{A \textit{ true} \quad \quad B \textit{ true}}{A \land B \textit{ true}}}$}} \quad& 
\begin{minipage}{2.5in}
{
\begin{Verbatim}[baselinestretch=.8,commandchars=\\\{\},codes={\catcode`$=3\catcode`^=7}]
[(\rtag{true} \bangclause{!(\rtag{ck} A-pf A)})
 (\rtag{true} \bangclause{!(\rtag{ck} B-pf B)})
 -->
 (\rtag{true} \huhclause{?(\rtag{ck} (\rtag{$\land$I} A-pf B-pf (\rtag{$\land$I} A B)) (\rtag{$\land$I} A B))})]
\end{Verbatim}
}
\end{minipage}
\end{tabular}
\caption{The introduction rule for $\land$ in ITT.}
\label{fig:itt}
\end{figure}

Figure~\ref{fig:itt} shows the introduction form for $\land$
(left) and its corresponding \slog{} transliteration (right). Checking
a derivation of $\land$I forces the checking of each sub-derivation,
and (upon success) populates the \rtag{true} relation with the
appropriate derivation.

Implication in ITT is managed by introducing, and then discharging,
assumptions: $A \supset B$ holds whenever $B$ may be proven by
assuming $A$. Figure~\ref{fig:impl} details the implication rule in
ITT (left) and \slog{} (right): the introduced hypothesis (named $u$
in $\supset{}\!I$) is subsequently discharged to produce an
assumption-free proof of $A \supset B~\textit{true}$. In \slog{}, the
rule for $\supset{}\!I$ introduces an assumption by forcing the
materialization of \texttt{(\rtag{assuming} A pf-B)}---other rules
then ``push down'' the assumption $A~\textit{true}$ to their eventual
uses (top right of Figure~\ref{fig:impl}), performing transitive
closure of assumptions to their usages in an on-demand fashion.

\begin{figure}[h!]
\begin{tabular}{c|c}
%% -- u
%% A  
%% ..
%% B 
%% -- 
%% A -> B

${\supset{}\!I}\quad \frac { \frac{}{\begin{array}{c}A~\textit{true}\\\vdots\\ B~\textit{true}\end{array}}~u } {\begin{array}{c}A \supset B~\textit{true}\end{array}}$

&
\begin{minipage}{2.5in}
{
\vspace{-.5in}
\begin{Verbatim}[baselinestretch=.8,commandchars=\\\{\},codes={\catcode`$=3\catcode`^=7}]
;; Propagate assumptions
(\rtag{true} \huhclause{?(\rtag{ck} (\rtag{assuming} P (\rtag{assumption} P)) P)} P)
...

[(\rtag{true} \bangclause{!(\rtag{ck} (\rtag{assuming} A pf-B) B)})
 -->
 (\rtag{true} \huhclause{?(\rtag{ck} (\rtag{$\supset$I} (\rtag{$\supset$} A B) pf-B) (\rtag{$\supset$} A B))})]
\end{Verbatim}
}
\end{minipage}
\end{tabular}
\caption{Implication in ITT and \slog{}}
\label{fig:impl}
\end{figure}


\paragraph*{First-Order Dependent Types: \lf{}}

\begin{figure}
\begin{displaymath}
\begin{tabular}{lrcllrcl}
\textit{\lf{} Contexts}& $\Gamma$ & $::=$ & $\varnothing$ &\textit{\lf{} Types} & $\tau \in T$ & $::=$& $X$ \\
&     &  $\mid$ & $\Gamma, x \! : \! T$ & && $\mid$& $\Pi x \! : \! T.~ T$ \\
&     &  $\mid$ & $\Gamma, x \! ::\!  K$ & &&$\mid$& $(T~ e)$ \\
\textit{\lf{} Kinds} & $K$ & $::=$ & $\Pi x \! : \! T. K \mid \ast$ \\ 
\end{tabular}
\end{displaymath}
\begin{tabular}{c|c}
%\toprule \\
{{\small\textsc{TA-Abs}}\quad\quad{\LARGE ${ \frac{\Gamma\, \vdash\, S\, ::\, \ast \quad \Gamma, \,x\, :\, S\, \vdash\, t\, : \,T}{\Gamma\, \vdash \, (\lambda (x\,:\,S) t) \,:\, \Pi x\, :\, S. T}}$}} \quad& 
\begin{minipage}{2.5in}
{\footnotesize
\begin{Verbatim}[baselinestretch=.8,commandchars=\\\{\},codes={\catcode`$=3\catcode`^=7}]
[(\rtag{::} \bangclause{!(\rtag{ck-k} $\Gamma$ S)} (\rtag{$\ast$}))
 (\rtag{:} \bangclause{!(\rtag{ck-t} (ext-env $\Gamma$ x S) t)} T)
 -->;------
 (\rtag{:} \bangclause{?(\rtag{ck-t} $\Gamma$ (\rtag{$\lambda$} x S t))} (\rtag{$\Pi$} x S T))]
\end{Verbatim}
}\end{minipage}
\\
\\
  {{\small\textsc{TA-App}}\quad\quad{\LARGE ${ \frac{\Gamma \, \vdash \, t_1 \,:\, \Pi x : S_1 . \, T \quad \Gamma \, \vdash \, t_2\, : \,S_2 \quad \Gamma\, \vdash\, S_1\, \equiv\, S_2}{\Gamma\, \vdash\, (t_1~t_2)\, : \,[ x \,\mapsto\, t_2 ] \, T}}$}}\quad&
\begin{minipage}{2.5in}
{\footnotesize
\begin{Verbatim}[baselinestretch=.8,commandchars=\\\{\},codes={\catcode`$=3\catcode`^=7}]
[(\rtag{:} \bangclause{!(\rtag{ck-t} $\Gamma$ t1)} (\rtag{$\Pi$} x S1 T))
 (\rtag{:} \bangclause{!(\rtag{ck-t} $\Gamma$ t2)} S2)
 (\rtag{true} \bangclause{!(\rtag{===} $\Gamma$ S1 S2)})
 -->;------
 (\rtag{:} \huhclause{?(\rtag{ck-t} $\Gamma$ (\rtag{app} t1 t2))}
   \{\rtag{subst} \bangclause{!(\rtag{do-subst} T x t2)}\})]
\end{Verbatim}
}\end{minipage} \\
\\
  {{\small\textsc{KA-App}}\quad\quad{\LARGE ${ \frac{\Gamma \, \vdash \, S \,::\, \Pi x : T_1 . \, K \quad \Gamma \, \vdash \, t\, : \,T_2 \quad \Gamma\, \vdash\, T_1\, \equiv\, T_2}{\Gamma\, \vdash\, (S~t)\, : \,[ x \,\mapsto\, t ] \, K}}$}}\quad&
\begin{minipage}{2.5in}
{\footnotesize
\begin{Verbatim}[baselinestretch=.8,commandchars=\\\{\},codes={\catcode`$=3\catcode`^=7}]
[(\rtag{::} \bangclause{!(\rtag{ch-k} $\Gamma$ S)} (\rtag{$\Pi$} x T1 K))
 (\rtag{:} \bangclause{!(\rtag{ck-t} $\Gamma$ t)} T2)
 (\rtag{true} \bangclause{!(\rtag{===} $\Gamma$ T1 T2)})
 -->;------
 (\rtag{:} \huhclause{?(\rtag{ch-t} $\Gamma$ (\rtag{type-app} S t))}
   \{subst !(do-subst K x t)\})]
\end{Verbatim}
}\end{minipage} \\
\\
%\bottomrule
\begin{minipage}{2.75in}
{\small
\begin{Verbatim}[baselinestretch=.8,commandchars=\\\{\},codes={\catcode`$=3\catcode`^=7}]
[(\rtag{->wh} \bangclause{!(\rtag{do->wh} t1)} t1')
 -->;------
 (\rtag{->wh} \huhclause{?(\rtag{do->wh} (\rtag{app} t1 t2))} (\rtag{app} t1' t2))]
\end{Verbatim}
}
\end{minipage}
&
\quad
\quad
\begin{minipage}{2.75in}
{\small
\begin{Verbatim}[baselinestretch=.8,commandchars=\\\{\},codes={\catcode`$=3\catcode`^=7}]
(\rtag{->wh} \huhclause{?(\rtag{do->wh} (\rtag{app} ($\lambda$ x T1 t1) t2))}
  \{subst \bangclause{!(do-subst t1 x t2)}\})
\end{Verbatim}
}\end{minipage} 
\end{tabular}

\caption{\lf{}: Syntax (top), selected rules (left) and \slog{} (right).}
\label{fig:lf}
\end{figure}

The Edinburgh Logical Framework (\lf{}) is a dependently-typed
$\lambda$-calculus~\cite{Harper:93}. It is a first-order dependent
type system, in the sense that it stratifies its objects into kinds, types
(families), and terms (values)---kinds may quantify over types, but
% Todo: Why put this idea here right up front?
not over other kinds. The syntax of \lf{} is detailed at the bottom of
Figure~\ref{fig:lf}---it extends STLC with kinds, which are either
$\ast$ or (type families) $\Pi x : T. K$, where $T$ is a simple type
(of kind $\ast$). \lf{} generalizes the arrow type to the dependent
product: $\Pi x: T. T$. System \lf{} enjoys several decidability
properties which make it particularly amenable to implementation in
\slog{}. The first is strong normalization, which implies that
reduction sequences for well-typed terms in our implementation will be
finite. The second is \lf{}'s focus on canonical forms and hereditary
substitution~\cite{harper:2007}. In \lf{}, terms are canonicalized to
weak-head normal form (WHNF); this choice enables inductive reasoning
on these canonical forms, and this methodology forms the basis for
% Todo: Say a bit more about Twelf? It's meant for theorem proving or programming?
Twelf~\cite{pfenning1998twelf}.

The \textit{judgments-as-types} principle interprets type checking for
\lf{} as proving theorems in intuitionstic predicate logic; Using this
principle, we may define traditional constructive connectives (such as
$\land$, $\lor$, and $\exists$) via type families and their associated
rules. For example, including in $\Gamma$ a binding $\land \mapsto
\Pi\, P : \textit{prop}.~ \Pi\, Q : \textit{prop}.~ \ast$ allows using
the constructor $\land$, though $\land$ must be instantiated with a
suitable $P$ and $Q$, which must necessarily be of some sort (e.g.,
\textit{prop}) also bound in $\Gamma$.

We have performed a transliteration of \lf{} as formalized in Chapter
2 of \textit{Advanced Topics in Types and Programming Languages
  (ATAPL)}~\cite{atapl}. Our transliteration (from pages 57--58)
consists of roughly 150 lines of \slog{}
code. Figure~\ref{fig:lf} details several of the key
rules. \textsc{TA-Abs} introduces a $\Pi$ type, generalizing the
\textsc{T-Abs} rule in Figure~\ref{fig:stlc}. The \textsc{TA-App}
applies a term $t_1$, of a dependent product type $\Pi x :S_1 ~.~T$,
whenever the input $t_2$ shares an equivalent type, $S_2$. The notion
of equality here is worth mentioning: $\equiv$ demands reduction of
its arguments to WHNF---\lf{} is constructed to identify terms under
WHNF, thus ensuring $\equiv$ will terminate as long as the term is
typeable. Reduction to WHNF is readily implemented in \slog{}; two
exemplary rules detailed at the bottom of Figure~\ref{fig:lf}
% Todo: I changed this from fig:lf-example; correct?
outline the key invariant in WHNF: reduce down the leftmost spine,
eliminating $\beta$-redexes via application. Equality checks are
demanded by the \textsc{TA-App} and \textsc{KA-App} rules, and force
normalization of their arguments to WHNF before comparison of
canonical forms, generating a witness in \rtag{true} before triggering
the head of the rule.


%% \begin{tabular}{c|c|c}
%% \begin{align}
%% \forall \, Q,&\, \forall\, P.& \\
%%   &\,Q \land P \rightarrow P \land Q
%% \end{align}
%% &
%% \begin{minipage}{2.75in}
%% {\small
%% \begin{Verbatim}[baselinestretch=.6,commandchars=\\\{\},codes={\catcode`$=3\catcode`^=7}]
%% (tcheck
%%  (extend
%%   (extend (bot) (var (prop)) (star))
%%   (var (land)) (PiK (x) (var (prop))
%%                     (PiK (y) (var (prop)) (star))))
%%  (lam (x) (var (prop))
%%       (lam (y) (var (prop))
%%            (type-app (var (land)) (ref (x))))))
%% \end{Verbatim}
%% }
%% \end{minipage}
%% &
%% \quad
%% \quad
%% \begin{minipage}{2.75in}
%% {\small
%% \begin{Verbatim}[baselinestretch=.6,commandchars=\\\{\},codes={\catcode`$=3\catcode`^=7}]
%% (tcheck
%%  (extend
%%   (extend (bot) (var (prop)) (star))
%%   (var (land)) (PiK (x) (var (prop))
%%                     (PiK (y) (var (prop)) (star))))
%%  (lam (x) (var (prop))
%%       (lam (y) (var (prop))
%%            (type-app (var (land)) (ref (x))))))
%% \end{Verbatim}
%% }\end{minipage} 
%% \end{tabular}




