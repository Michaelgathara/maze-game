\section{Structurally Recursive Datalog}
\label{sec:semantics}
%\lstset{language=Slog}
%
The core semantic difference between \slog{} and Datalog is to allow
structurally recursive, first-class facts.  This relatively minor
semantic change enables both enhanced expressivity (naturally
supporting a wide range of Turing-equivalent idioms, as we demonstrate
in Section~\ref{sec:apps}) and anticipates compilation to parallel
relational algebra (which interns all facts and distributes facts via
their intern key). In this section, we present the formal semantics of
a language we call Structurally Recursive Datalog (henceforth
\core{}), the core language extending Datalog to which \slog{}
programs compile. All of the definitions related to \core{} have been
formalized, and all of the lemmas and theorems presented in this
section have been formally proven in Isabelle.

\paragraph*{Syntax}
%
\begin{wrapfigure}{r}{6cm}
\vspace{-0.5cm}
\begin{grammar}
  <Prog>   ::= <Rule>*
  
  <Rule>   ::= <Clause> $\leftarrow$ <Clause>*
  
  <Clause> ::= (tag <Subcl>*)
  
  <Subcl>  ::= (tag <Subcl>*) | <Var> | <Lit>
  
  <Lit>    ::= <Number> | <String> | ...
\end{grammar}
\caption{Syntax of \core{}: \textit{tag} is a relation name.}
\label{fig:dls-syntax}
\end{wrapfigure}
%
The syntax of \core{} is shown in Figure~\ref{fig:dls-syntax}. As in
Datalog, a \core{} program is a collection of Horn clauses. Each
rule $R$ contains a set of body clauses and a head clause, denoted by
$\textit{Body}(R)$ and $\textit{Head}(R)$ respectively. \core{} (and
\slog{}) programs must also be well-scoped: variables appearing in a
head clause must also be contained in the body.
  
We define a strict syntactic subset of \core{}, \textit{DL} as the
restriction of \core{} to clauses whose arguments are literals (i.e.,
${\langle\textit{Subcl}\rangle}_{\textit{DL}} ::= \langle\textit{Var}\rangle \mid
\langle\textit{Lit}\rangle$). This subset (and its semantics) corresponds to
Datalog.

\paragraph*{Fixed-Point Semantics}

The fixed-point semantics of a \core{} program $P$ is given via the
least fixed point of an \emph{immediate consequence} operator
$\textit{IC}_P : \textit{DB} \rightarrow \textit{DB}$. Intuitively,
this immediate consequence operator derives all of the immediate
implications of the set of rules in $P$. A
database $\textit{db}$ is a set of facts ($\textit{db} \in \textit{DB}
= \mathcal{P}(\textit{Fact})$). A fact is a clause without variables:
\begin{align*}
\textit{Fact} &::= (\textit{tag}\ \textit{Val}^*) & \quad
\textit{Val}  &::= (\textit{tag}\ \textit{Val}^*)\ \mid \textit{Lit}
\end{align*}

In Datalog, $\textit{Val}$s are restricted to a finite set of atoms
($\textit{Val}_{\textit{DL}} ::= \textit{Lit}$). To define $IC_P$, we
first define the immediate consequence of a rule $IC_R : DB
\rightarrow DB$, which supplements the provided database with all the
facts that can be derived directly from the rule given the available
facts in the database:

\begin{align*}    
\textit{IC}_R(db) \triangleq\ & db\ \cup \bigcup \bigl\{ \textit{unroll}(\textit{Head}(R)\big[\overrightarrow{v_i\backslash x_i}\big]) | \\
& \{\overrightarrow{x_i \rightarrow v_i}\} \subseteq (\textit{Var} \times \textit{Val})\ \wedge
\textit{Body}(R)\big[\overrightarrow{v_i \backslash x_i}\big] \subseteq db \bigr\}
\end{align*}

The $unroll$ function has the following definition:

\begin{centering}
  {\centering$\begin{array}{rcl}
\quad\quad\quad \textit{unroll}\bigl((\textit{tag}\ \textit{item}_1\ ...\ \textit{item}_n)\bigr) &\triangleq&
\{(\textit{tag}\ \textit{item}_1\ ...\ \textit{item}_n)\} \cup \bigcup_{i \in {1 ... n}} \textit{unroll}(\textit{item}_i)   \\
\textit{unroll}(v)_{v \in \textit{Lit}} &\triangleq& \{\}
    \end{array}$}
\end{centering}

The purpose of the unroll function is to ensure that all nested facts are included in the database as well, a property we call \emph{subfact-closure}. This property is crucial to the semantics of \core{} (and \slog{}), because in \core{}, each nested fact is a fact in own right, and not merely a carrier of structured data. Later sections (starting in section~\ref{sec:semantics:extensions}) illustrate the importance of subfact closure by demonstrating how  we utilize this behavior to construct idioms that make programming in \slog{} more convenient.

The immediate consequence of a program is the union of the immediate
consequence of each of its constituent rules, $
\textit{IC}_P(\textit{db}) \triangleq \textit{db} \cup \bigcup_{R \in
  P} \textit{IC}_R(\textit{db})$. Observe that $\textit{IC}_P$ is
monotonic over the the lattice of databases whose bottom element is
the empty database. Therefore, if $\textit{IC}_P$ has any fixed
points, it also has a least fixed point \cite{tarski1955lattice}.
Iterating to this least fixed point directly gives us a na\"ive, incomputable
fixed-point semantics for \core{} programs.
Unlike pure Datalog, existence of a finite fixed point is not guaranteed in
\core{}. This is indeed a reflection of the fact that \core{} is
Turing-complete. The \core{} programs whose immediate consequence
operators have no finite fixed points are non-terminating.

As discussed earlier, all \slog{} databases must be subfact-closed (i.e. all subfacts are first-class facts). We can show that the least fixed point of the immediate consequence operator has the property that it is subfact-closed.
\begin{lemma}
(Formalized in Isabelle.) The least fixed point of $\textit{IC}_P$ is subfact-closed. 
\end{lemma}

It is worth pointing out that the fixed point semantics of Datalog is similar, the only difference being that the $\textit{unroll}$ function is not required, as Datalog clauses do not contain subclauses.

\paragraph*{Model Theoretic Semantics}

The model theoretic semantics of \core{} closely follows the model
theoretic semantics of Datalog, as presented in, e.g.,
\cite{ceri1989you-datalog}. The \emph{Herbrand universe} of a \core{}
program is the set of all of the facts that can be constructed from
the relation symbols appearing in the program. Because \core{} facts
can be nested, the Herbrand universe of any nontrivial \core{} program
is infinite. One could for example represent natural numbers in
\core{} using the zero-arity relation \lstinline{Zero} and the unary
relation \lstinline{Succ}. The Herbrand universe produced by just
these two relations, one zero arity and one unary, is inductively infinite.

A \emph{Herbrand Interpretation} of a \core{} program is any subset of its Herbrand universe that is subfact-closed. In other words, if $I$ is a Herbrand Interpretation, then $I = \bigcup \{\textit{unroll}(f) \linebreak |\ f \in I\}$. For Datalog, the Herbrand Interpretation is defined similarly, with the difference that subfact-closure is not a requirement for Datalog, as Datalog facts do not contain subfacts.

Given a Herbrand Interpretation $I$ of a \core{} program $P$, and a rule $R$ in $P$, we say that $R$ is true in $I$ ($I \models R$) iff for every substitution of variables in $R$ with facts in $I$, if all the body clauses with those substitutions are in $I$, so is the head clause of $R$ with the same substitutions of variables.
\begin{align*}  
& I \models R\ ~\textrm{iff} ~
 \forall \{\overrightarrow{x_i \rightarrow v_i}\}\ .\ \textit{Body}(R)\big[\overrightarrow{v_i\backslash x_i}\big] \subseteq I \longrightarrow \textit{Head}(R)\big[\overrightarrow{v_i\backslash x_i}\big] \in I 
\end{align*}

If every rule in $P$ is true in $I$, then $I$ is a \emph{Herbrand model} for $P$. The denotation of $P$ is the intersection of all Herbrand models of $P$. We define $\mathbf{M}(P)$ to be the set of all Herbrand models of $P$, and $D(P)$ to be the denotation of $P$. We then have $D(P) \triangleq \!\!\!\!\!\!\bigcap\limits_{I\in\ \mathbf{M}(P)}\!\!\!\!\! I$. It can be shown that such an intersection is a Herbrand model itself:
\begin{lemma}
The intersection of a set of Herbrand models is also a Herbrand model.
\end{lemma}

Unlike Datalog, nontrivial \core{} programs have Herbrand universes
that are infinite. Thus, a \core{} program may have only infinite
Herbrand models. If a \core{} program has no finite Herbrand models,
its denotation is infinite and so no fixed-point may be finitely
calculated using the fixed-point semantics. We now relate the operational
semantics of \core{} to its model-theoretic semantics.

\paragraph*{Equivalence of Model-Theoretic and Fixed-Point Semantics}

To show that the model-theoretic and fixed-point semantics of \core{}
compute the same Herbrand model, we need to show that the least fixed
point of the immediate consequence operator is equal to the
intersection of all the Herbrand models for any program. We start by
proving the following lemmas (proved in Isabelle; proofs elided for
space).

\begin{lemma}
Herbrand models of a \core{} program are fixed points of the immediate consequence operator.    
\end{lemma}

\begin{lemma}
Fixed points of the immediate consequence operator of a \core{} program that are subfact-closed are Herbrand models of the program.    
\end{lemma}

\begin{wrapfigure}{l}{6.75cm}
%\begin{tabular}{ c c }

\setlength{\grammarindent}{7em} % increase separation between LHS/RHS 
% \begin{minipage}{.50\textwidth}
\begin{grammar}
<toplvl-rule> ::= <rule> | <hclause> 

<rule> ::= "[" <hd-item>* "<--" <bd-item>* "]"
      \alt "[" <bd-item>* "-->" <hd-item>* "]" 

<bd-item> ::= <rule> | <bclause> 

<hd-item> ::= <rule> | <hclause> 

<bclause> ::= "("<tag> <ibclause>*")"  
         \alt "(=" <var> "("<tag> <ibclause>*"))"

<hclause> ::= "("<tag> <ihclause>*")" 
         \alt "(=" <var> "("<tag> <ihclause>*"))"

<atom> ::= <var> | <lit>

<lit> ::= <string> | <number>

% \end{grammar}
% \end{minipage}
% \begin{minipage}{.45\textwidth}
% \begin{grammar}
<ihclause> ::= "("<tag> <ihclause>*")" 
          \alt "?("<tag> <ibclause>*")"
          \alt "{"<tag> <ibclause>*"}" 
          \alt "["<hlist-item>*"]" 
          \alt "?["<blist-item>*"]"
          \alt <atom>

<ibclause> ::= "("<tag> <ibclause>*")" 
          \alt "!("<tag> <ihclause>*")" 
          \alt "{"<tag> <ibclause>*"}" 
          \alt "["<blist-item>*"]" 
          \alt "!["<hlist-item>*"]"
          \alt <atom>

% TODO remove list-related grammar if lists are taken out of the prose.
<hlist-item> ::= <ihclause> | <ihclause> "…"

<blist-item> ::= <ibclause> | <ibclause> "…"
\end{grammar}
% \end{minipage}

%&
%\\
%\end{tabular}
\caption{The syntax of \slog{}. \synt{var} is the set of variables, and \synt{tag} is the set of relation names. A few syntactic forms, including disjunction, have been elided.}
\label{fig:syntax}
\vspace{-1.5cm}
\end{wrapfigure}
%

By proving that the Herbrand models and subfact-closed fixed points of
the immediate consequence operator are the same, we conclude that the
least fixed point of the immediate consequence operator
$\textit{IC}_P$ (a subfact-closed database) is equal to the
intersection of all its Herbrand models.

\begin{theorem}
The model theoretic semantics and fixed point semantics of \core{} are equivalent.
\end{theorem}

Proof sketch: Form the lemma that all Herbrand models are fixed points of $\textit{IC}_P$, we conclude that $D(P)$ is a superset of the intersection of all the fixed points. We know that the least fixed point of $\textit{IC}_P$ (which we'll call $\textit{LFP}_P$) is a subset of the intersection of all the fixed points. We therefore have $\textit{LFP}_P \subseteq D(P)$. From the fact the $\textit{LFP}_P$ is a Herbrand model, we conclude that $D(P) \subseteq \textit{LFP}_P$. Putting these facts together, we conclude that $\textit{LFP}_P = D(P)$.
% From these lemmas, we know that Herbrand models and fixed points of the immediate consequence operator are the same. We therefore need to show that the intersection of all the fixed points of the immediate consequence operator is the least fixed point of the operator. From the definition of the least fixed point, it immediately follows that if $db_1$ and $db_2$ are fixed points of the operator, $IC_P(db_1 \cap db_2) \subseteq db_1 \cap db_2$. Conversely, from the definition of $IC_P$, it follows that for all $db$, $db \subseteq IC_P(db)$. Putting these two facts together, we have: for all fixed points $db_1$ and $db_2$ of $IC_P$, $IC_P(db_1 \cap db_2) = db_1 \cap db_2$.


\renewcommand{\multicolsep}{3pt plus 1pt minus 1pt}

% https://tex.stackexchange.com/questions/24886/which-package-can-be-used-to-write-bnf-grammars
% https://mirrors.rit.edu/CTAN/macros/latex/contrib/mdwtools/syntax.pdf
\renewcommand{\ulitleft}{\bf\ttfamily\frenchspacing}
\renewcommand{\ulitright}{}

\subsection{Key extensions to the core language}
\label{sec:semantics:extensions}
%
With subfacts, a common idiom becomes for a subfact to appear in the body of a rule, while its surrounding fact and any associated values are meant to appear in the head. For these cases, we use a ? clause, an s-expression marked with a ``?'' at the front to indicate that although it may appear to be a head clause, it is actually a body clause and the rule does not fire without this fact present to trigger it. The following rule says that if a \texttt{(\rtag{ref} x)} AST exists, then x is a free variable with respect to it.
%
\begin{Verbatim}[baselinestretch=0.75,commandchars=\\\{\}]
(\rtag{free} \huhclause{?(\rtag{ref} x)} x)
\end{Verbatim}
%
which desugars to the rule
%
\begin{Verbatim}[baselinestretch=0.75,commandchars=\\\{\}]
[(= e-id (\rtag{ref} x)) --> (\rtag{free} e-id x)]
\end{Verbatim}
%
exposing that the ? clause is an implicit body clause. But if there are
no body clauses apart from the ? clauses, the rule may be written without
square braces and an arrow to show direction.

\begin{wrapfigure}{r}{6cm}
\vspace{-0.25cm}
\begin{Verbatim}[baselinestretch=0.75,commandchars=\\\{\}]
[(=/= x y) (\rtag{free} Eb y)
 --> (\rtag{free} \huhclause{?(\rtag{lam} x Eb)} y)]
[(or (\rtag{free} Ef x) (\rtag{free} Ea x))
 --> (\rtag{free} \huhclause{?(\rtag{app} Ef Ea)} x)
\end{Verbatim}
\end{wrapfigure}
%
Two more rules are needed to define a free-variables analysis.
%
The second of these shows another extension: disjunction in the body of
a rule is pulled to the top level and splits the rule into multiple rules.
In this case, there is both a rule saying that a free variable in \texttt{Ef}
is free in the application and a rule saying that a free variable in \texttt{Ea}
is free in the application.

Another core mechanism in \slog{} is to put head clauses in position where a body clause is expected.
%
Especially because an inner clause can be \emph{responded to} by a fact surrounding it, or by rules producing that fact,
being able to emit a fact on-the-way to computing a larger rule is what permits natural-deduction-style rules through a kind of rule
splitting, closely related to continuation-passing-style (CPS) conversion~\cite{appel2007compiling}.
A ! clause, under a ? clause or otherwise in the position of a body clause,
is a clause that will be deduced as the surrounding rule is evaluated, so long as any ? clauses are satisfied and any subexpressions
are ground (any clauses it depends on have been matched already). These ! clauses are intermediate head clauses; technically the head clauses of subrules, which they are compiled into internally. 

Consider the example in Figure~\ref{fig:natural-deduction-plus}, which lets us prove an arithmetic statement like\newline\texttt{(plus (plus (nat 1) (nat 2)) (nat 1)) $\Downarrow$ 4}.
We can construe this rule in a few ways, as written. It could be that both the expression and value should be provided and are proved according to these rules, or it could be treated as a calculator, with the expression provided as input.

\vspace{-0.4cm}
\begin{figure*}[h]
\begin{multicols}{2}
\begin{Verbatim}[baselinestretch=0.75,commandchars=\\\{\}]
(\rtag{interp} \huhclause{?(\rtag{do-interp} (\rtag{nat} n))} n)

  
[(\rtag{interp} \bangclause{!(\rtag{do-interp} e0)} v0)
 (\rtag{interp} \bangclause{!(\rtag{do-interp} e1)} v1)
 (+ v0 v1 v) 
 --> \comm{;-------- [plus]}
 (\rtag{interp} \huhclause{?(\rtag{do-interp} (\rtag{plus} e0 e1))} v)]
\end{Verbatim}
\columnbreak
\[
  \frac{\ }{\texttt{(nat $n$)} \Downarrow n}\text{\small[nat]}
\]
  \vspace{0.75cm}
\[
 \frac{e_0 \Downarrow v_0 \hspace{1cm} e_1 \Downarrow v_1 \hspace{1cm} v=v_0+v_1}{(\texttt{plus}\ e_0\ e_1) \Downarrow v} \text{\small[plus]}
\]
\end{multicols}
\caption{Natural-deduction-style reasoning with ! clauses in \slog{}.}
\label{fig:natural-deduction-plus}
\end{figure*}
\vspace{-0.4cm}

\begin{wrapfigure}{r}{5.75cm}
\vspace{-0.3cm}
\begin{Verbatim}[baselinestretch=0.75,commandchars=\\\{\}]

[(\rtag{interp} \bangclause{!(\rtag{do-interp} e0)} v0)
 (\rtag{interp} \bangclause{!(\rtag{do-interp} e1)} v1) 
 --> \comm{;-------- [plus]}
 (\rtag{interp} \huhclause{?(\rtag{do-interp} (\rtag{plus} e0 e1))}
         \{+ v0 v1\})]




(\rtag{append} \huhclause{?(\rtag{do-append} [] ls)} ls)

[(\rtag{append} \bangclause{!(\rtag{do-append} ls0 ls1)} ls')
 -->
 (\rtag{append} \huhclause{?(\rtag{do-append} [x ls0 ...] ls1)}
         [x ls' ...])]

\comm{; or ind. case could even be written:}
(\rtag{append} \huhclause{?(\rtag{do-append} [x ls0 ...] ls1)}
        [x
         \{\rtag{append} \bangclause{!(\rtag{do-append} ls0 ls1)}\}
         ...])
\end{Verbatim}
\end{wrapfigure}
%
Subclauses, written with parentheses, are treated as top-level clauses whose id column value is unified at the position of the subclause. Another common use for a relation is as a function, or with a designated output column, deterministic or not, so \slog{} also supports this type of access via \{ \!\} inner clauses, which have their final-column value unified at the position of the curly-brace subclause. For example, the rule in Figure~\ref{fig:natural-deduction-plus} could also have been written as below, with the clause \texttt{\{+ v0 v1\}} in place of variable \texttt{v}.
This example illustrates that this syntax can also be used for built-in relations like \texttt{+}.

Putting this all together and adding \slog{}'s built-in list syntax---currently implemented as linked-lists of \slog{} facts in the natural way---we can implement rules for appending lists, a naturally direct-recursive task due to a linked list naturally having its first element at its front, so a second list can only be appended to the back of the first list, and the front element onto the front of that.
%






