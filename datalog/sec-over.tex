\section{Slog: Declarative Parallel Deduction of Structured Data}
\label{sec:over}
%
For Datalogs used in program analysis, manipulation of abstract syntax trees (ASTs) is among the most routine tasks.
%
Normally, to provide such ASTs as input to a modern Datalog engine, one first requires an external flattening tool that walks the richly structured syntax tree and produces a stream of flat, first-order facts to be provided as an input database. For example, the Datalog-based Java-analysis framework DOOP~\cite{Bravenboer:2009, 10.1145/1639949.1640108}, ported to Souffl\'e~\cite{antoniadis2017porting} in 2017, has a substantial preprocessor (written in Java) to be run on a target JAR to produce an input database of AST facts for analysis.

A key observation that initially motivated our work into this subject was that although this preparatory
transformation is required to provide an AST as a database of first-order facts,
the same work could not be done from within these Datalogs because it required
generating unique identities (i.e., pointers to intern values) for inductively defined terms. In fact, any work generating ASTs as facts can not be
done within Datalog itself but must be an extension to the language. Consider the pair of nested expressions
that form an identity function:

%\vspace{-0.65cm}
\begin{multicols}{3}
  \colorbox{white}{\texttt{(\rtag{lam} "x" (\rtag{ref} "x"))}}

  \hspace{0.5cm} $\overset{\text{flattens}}{\longrightarrow}$ %\hspace{-0.5cm}
  
  \begin{minipage}{\linewidth}
  \texttt{(= lam-id (\rtag{lam} "x" ref-id))}
  
  \texttt{(= ref-id (\rtag{ref} "x"))}
  \end{minipage}
\end{multicols}

Supplying unique intern values \texttt{lam-id} and \texttt{ref-id} as an extra column for those relations, and thus permitting them to be linked together, is the substance of this preparatory transformation. Our language, \slog{}, proposes this interning behavior for facts be ubiquitous, accounted for at every iteration of relational algebra used to implement the underlying HornSAT fixed point. 

In Souffl\'e, the language has more recently provided abstract data-type (ADT) declarations and struct/record types for heap-allocated values which can be built up into ASTs
or other such structured data. These datatypes must be declared and can then be used as \$ expressions within rules; e.g., \texttt{\$lam("x",\$ref("x"))}.
The downside of these ADTs in Souffl\'e is that they are not treated as facts for the purposes of triggering rules and are
not indexed as facts, which would permit more efficient access patterns.

Our language, \slog{}, respects a \emph{subfact closure property}: every subfact is itself a first-class fact in the language and every top-level fact (and subfact) is a first-class value and has a unique identity (as an automatic column-$0$ value added to the relation). A clause \texttt{(\rtag{foo} x y)} in \slog{}, always has an implied identity column and is interpreted the same as \texttt{(= \_ (\rtag{foo} x y))} if it's missing (where underscore is a wildcard variable). A nested pair of linked facts like \texttt{(\rtag{foo} x (\rtag{bar} y) z)} is desugared as \texttt{(= \_ (\rtag{foo} x id z))} and \texttt{(= id (\rtag{bar} y))}. Thus we can represent an identity function's AST in \slog{} as the directly nested fact and subfact \texttt{(\rtag{lam} "x" (\rtag{ref} "x"))}; under the hood this will be equivalent to two flat facts with a 0-column id provided by an interning process that occurs at the discovery of each new \slog{} fact.

\begin{itemize}
\item In \slog{}, each structurally unique fact/subfact has a unique intern-id stored in its 0 column so it may be referenced as another fact's subfact and treated as a first-class value.
\item In \slog{}, all data is at once a first-class fact (able to trigger rule evaluation), a first-class value (able to be referenced by other facts/values), and a first-class thread of execution (treated uniformly by a data-parallel MPI backend that dynamically distributes the workload spatially within, and temporally across, fixed-point iterations).
\end{itemize}

With subfacts as first-class citizens of the language (see Section~\ref{sec:semantics} for details), able to trigger rules, various useful idioms emerge in which a subfact triggers a response from another rule via an enclosing fact (see Section~\ref{sec:semantics:extensions} for extensions and idioms). Using these straightforward syntactic extensions enables a wide range of deduction and reasoning systems (see Section~\ref{sec:apps} for a discussion of applications in program analyses and type systems). Because subfacts are first-class in \slog{}, rules that use them will naturally force the compiler to include appropriate indices enabling efficient access patterns, and represent thread joins in the natural data-parallelism \slog{} exposes. As a result, we are able to show a deep algorithmic improvement and parallelism over current state-of-art systems in the implementations of analyses we generate (see Section~\ref{sec:eval} for our evaluation with apples-to-apples comparisons against Souffl\'e and RadLog). In some experiments, \slog{} finishes in 4--8 seconds with Souffl\'e taking 1--3 hours---attesting to the importance of subfact indices. In others, we observe efficient strong-scaling on up to hundreds of threads, showing the value of our data-parallel backend. 



