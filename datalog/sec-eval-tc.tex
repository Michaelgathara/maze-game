\subsection{Transitive Closure}
\label{sec:eval:tc}
\label{sec:eval:pb1}

We sought to compare \slog{}'s full-system throughput on vanilla
Datalog against two comparable production systems: Souffl\'e and
Radlog. Souffl\'e is engineered to achieve the best-known performance
on unified-memory architectures, and supports parallelism via
OpenMP. Radlog is a Hadoop-based successor to the BigDatalog deductive
inference system, which uses Apache Spark to perform distributed joins
at scale~\cite{rasql}. We originally sought to compare \slog{}
directly against BigDatalog, but found it does not support recent
versions of either Spark or Java (being built to target Java
1.5). Under direction of BigDatalog's authors, we instead used Radlog,
which is currently under active development and runs on current
versions of Apache Spark.

We performed comparisons on an \textsf{Standard\_M128s} instance
rented from Microsoft Azure~\cite{mseries}. The node used in our
experiments has 64 physical cores (128 threads) running Intel Xeon
processors with a base clock speed of 2.7GHz and 2,048GB of RAM. To
directly compare \slog{}, Souffl\'e, and Radlog, we ran each on the
same machine using 15, 30, 60, and 120 threads. We ran \slog{} using
OpenMPI version 4.1.1 and controlled core counts via
\texttt{mpirun}. We compiled Souffl\'e from its Git repository, using
Souffl\'e's compiled mode to compile each benchmark separately to use
the requisite number of threads before execution. Radlog runs natively
on Apache Spark, which subsequently runs on Hadoop. To achieve a fair
comparison against Souffl\'e and \slog{}, we ran Radlog using Apache
Spark configured in local mode; Spark's local mode circumvents the
network stack and runs the application directly in the JVM. We used
three large graphs shown the first column of Table~\ref{tab:single-results}:
\textsc{fb-media} is media-related pages on Facebook,
\textsc{ring10000} is ring graph of 10,000 nodes, and
\textsc{suitesparse} is from the UF Sparse Matrix
Collection~\cite{Davis:2011:UFS:2049662.2049663}. We configured Radlog
according the directions on its website, experimenting with a variety
of partitions (used for shuffling data between phases) to achieve the
best performance we could. Ultimately, we used three times as many
partitions as available threads, except for \textsc{ring10000}, for
which we found higher partition counts caused significantly lower
performance.

\begin{table}
\centering
\caption{Single-node TC Experiments}
\begin{tabular}{m{1.7cm}m{1cm}m{2cm}|cm{.5cm}m{.5cm}m{.5cm}m{.5cm}}
\toprule
\multicolumn{3}{c}{Graph Properties}&& \multicolumn{4}{c}{Time (s) at Process Count} \\
Name & Edges & {\quad$\mid$TC$\mid$} & System & 15 & 30 & 60 & 120 \\
\midrule
\multirow{3}{*}{\textsc{fb-media}} & \multirow{3}{*}{206k} & \multirow{3}{*}{96,652,228} & \slog & 62 & 40 & 21 & \textbf{18} \\
 &&& Souffl\'e & 35 & 33 & 34 & 37 \\
 &&& Radlog & 254 & 295 & 340 & 164 \\
\midrule \midrule
\multirow{3}{*}{\textsc{ring10000}} & \multirow{3}{*}{10k} & \multirow{3}{*}{100,020,001} & \slog & 363 & 218 & 177 & \textbf{115} \\
 &&& Souffl\'e & 149 & 143 & 140 & 141 \\
 &&& Radlog & 464 & 646 & 852 & 1292 \\
\midrule \midrule
\multirow{3}{*}{\textsc{suitesparse}} & \multirow{3}{*}{412k} & \multirow{3}{*}{3,354,219,810} & \slog & -- & 1,593 & 908 & \textbf{671} \\
 &&& Souffl\'e & 1,417 & 1,349 & 1,306 & 1,282 \\
 &&& Radlog & -- & -- & -- & -- \\
\bottomrule
\end{tabular}
\label{tab:single-results}
\end{table}

Table~\ref{tab:single-results} details the results of our single-node
performance comparisons in seconds for each thread count, where each
datapoint represents the best of three runs (lower is
better). Experiments were cut off after 30 minutes. In every case, we
found that \slog{} produced the best performance overall at 120
threads, even compared to Souffl\'e's best time. However, as expected,
our results indicate that Souffl\'e outperforms \slog{} at lower core
counts (below 60). Souffl\'e implements joins with tight loops in
\CC{}, and (coupled with its superior single-node datastructures) this
allows Souffl\'e to achieve better performance than either \slog{} or
Radlog at lower core counts. We found that Radlog did not scale nearly
as well as either Souffl\'e or \slog{}. We expected this would be the
case: both \slog{} and Souffl\'e compile to \CC{}. By comparison,
Radlog's Spark-based architecture incurs significant sequential
overhead due to the fact that it is implemented on top of the JVM and
pays a per-iteration penalty by using Hadoop's aggregation and
shuffling phase. \slog{} also incurs sequential overhead compared to
Souffl\'e due to its distributing results after every iteration,
though results indicate that our MPI-based implementation helps
ameliorate this compared to Radlog.



