\subsection{Type Systems}
\label{sec:apps:ts}

Along with operational semantics and program analyses, \slog{}
naturally enables the realization of structural type systems based on
constructive logics~\cite{tapl,atapl}.  These systems are often
specified via an inductively-defined typing judgment, whose
derivations may be represented in \slog{} via (sub)facts and whose
typing rules may be realized as rules in \slog{} (providing their
evaluation may be operationalized via \slog{}'s idioms). For example,
the rules for the simply-typed $\lambda$-calculus (STLC) in
Figure~\ref{fig:stlc} define the judgment $\Gamma \vdash e
:\tau$---under typing environment $\Gamma$, $e$ has been proven to
have type $\tau$. Proofs of this judgment are represented via \slog{}
facts of the form \texttt{(\rtag{:} (\rtag{ck} $\Gamma$ e) T)}; each
rule in the type system is then mirrored by a corresponding rule in
\slog{}.

When implementing a type system in \slog{}, it is crucial to consider
some important differences between \slog{} and natural deduction
per-se. First, equivalence in \slog{} is intensional, via fact
interning (as in type theories such as Coq's Calculus of Inductive
Constructions). For example, while our type checker for STLC decides
$\Gamma \vdash e :\tau$ via structural recursion on $e$, there are
infinitely many $\Gamma' \supseteq \Gamma$ for which $\Gamma' \vdash e
:\tau$ also holds---materializing these (infinite) $\Gamma'$
would result in nontermination.

Here, we focus on the presentation of algorithmic (i.e.,
syntax-directed) type checking procedures. The decidability of these
systems follows immediately from their structurally-recursive nature,
a property inherited by their \slog{} counterparts. We expect
enumerating terms in other theories which enjoy strong normalization
will readily follow. We anticipate \slog{}'s declarative style may
also be a natural fit for type synthesis, by bounding (potentially
infinite) rewritings using a decreasing ``fuel'' parameter. However,
we leave this, along with explorations of other (bidirectional,
substructrural, etc...)  type systems in \slog{} to future work.

\paragraph*{Simply-typed $\lambda$-calculus}

\begin{figure}
\begin{displaymath}
\begin{tabular}{lrcllrcl}
\textit{STLC Terms}& $e$ & $::=$ & $\big(\lambda (x\!:\!\tau)\,e\big)$ & \textit{STLC Types}& $\tau \in T$ & $::=$& $\tau \rightarrow \tau$ \\
&     &  $\mid$ & $(e_0 ~ e_1)$ & && $\mid$& \textsf{nat} \\
&     &  $\mid$ & $x$ & & & $\mid$ & $...$ \\
\end{tabular}
\end{displaymath}

\begin{flushleft}
\fbox{$\Gamma \vdash e : \tau$}
\end{flushleft}
\begin{tabular}{c|c}
%\toprule \\
 {{\small\textsc{T-Var}}\quad\quad{\LARGE ${ \frac{x : T \, \in\, \Gamma}{\Gamma\, \vdash\, x : T}}$}} \quad&
\begin{minipage}{2.5in}
\begin{Verbatim}[baselinestretch=.8,commandchars=\\\{\},codes={\catcode`$=3\catcode`^=7}]
[-->;-------- T-Var \\
 (\rtag{:} \huhclause{?(\rtag{ck} $\Gamma$ (\rtag{ref} x))} \{\rtag{env-map} $\Gamma$ x\})]
\end{Verbatim}
\end{minipage}
\\
\\
{{\small\textsc{T-Abs}}\quad\quad{\LARGE ${ \frac{\Gamma, x\, : \,T_1 \, \vdash \, e \, : \, T_2}{(\lambda\,(x\,:\, T_1)\,e)\, : T_1 \rightarrow \,T_2}}$}} \quad& 
\begin{minipage}{2.5in}
\begin{Verbatim}[baselinestretch=.8,commandchars=\\\{\},codes={\catcode`$=3\catcode`^=7}]
[(\rtag{:} \bangclause{!(\rtag{ck} (\rtag{ext-env} $\Gamma$ x T1) e)} T2)
 -->;-------- T-Abs
 (\rtag{:} \huhclause{?(\rtag{ck} $\Gamma$ (\rtag{$\lambda$} x T1 e))} (\rtag{->} T1 T2))]
\end{Verbatim}
\end{minipage}
\\
\\
 
{{\small\textsc{T-App}}\quad\quad{\LARGE ${ \frac{{\Gamma \, \vdash \, e_0\,:\,T_0\, \rightarrow\, T_1}\quad e_1\,:\,T_0}{\Gamma \, \vdash \, (e_0~e_1)\,:\,T_1}}$ }}\quad& 
\begin{minipage}{2.5in}
\begin{Verbatim}[baselinestretch=.8,commandchars=\\\{\},codes={\catcode`$=3\catcode`^=7}]
[(\rtag{:} \bangclause{!(ck $\Gamma$ e0)} (\rtag{->} T0 T1))
 (\rtag{:} \bangclause{!(ck $\Gamma$ e1)} T0)
 -->;-------- T-App
 (\rtag{:} \huhclause{?(\rtag{ck} $\Gamma$ (\rtag{app} e0 e1))} T1)]
\end{Verbatim}
\end{minipage} \\
%\bottomrule
\end{tabular}

\caption{Syntax (top) and semantics (left) of STLC; equivalent \slog{} (right).}
\label{fig:stlc}
\end{figure}

%% \begin{figure}
%% \begin{displaymath}
%% \begin{tabular}{lrcllrcl}
%% \textit{STLC Terms}& $e$ & $::=$ & $\big(\lambda (x\!:\!\tau)\,e\big)$ & \textit{STLC Types}& $\tau \in T$ & $::=$& $\tau \rightarrow \tau$ \\
%% &     &  $\mid$ & $(e_0 ~ e_1)$ & && $\mid$& $...$ \\
%% &     &  $\mid$ & $x$ & & &  \\
%% \textit{\lf{} Contexts}& $\Gamma$ & $::=$ & $\varnothing$ &\textit{\lf{} Types} & $\tau \in T$ & $::=$& $X$ \\
%% &     &  $\mid$ & $\Gamma, x \! : \! T$ & && $\mid$& $\Pi x \! : \! T.~ T$ \\
%% &     &  $\mid$ & $\Gamma, x \! ::\!  K$ & &&$\mid$& $(T~ e)$ \\
%% \textit{\lf{} Kinds} & $K$ & $::=$ & $\Pi x \! : \! T. K \mid \ast$ \\ 
%% \end{tabular}
%% \end{displaymath}
%% \caption{Syntax of STLC and \lf{}.}
%% \label{fig:syntax}
%% \end{figure}


%% \begin{figure}
%% \begin{flushleft}
%% \fbox{$\Gamma \vdash e : \tau$}
%% \end{flushleft}
%% \begin{tabular}{c|c}
%% %\toprule \\
%%  {{\small\textsc{T-Var}}\quad\quad{\LARGE ${ \frac{x : T \, \in\, \Gamma}{\Gamma\, \vdash\, x : T}}$}} \quad&
%% \begin{minipage}{2.5in}
%% \begin{Verbatim}[baselinestretch=.8,commandchars=\\\{\},codes={\catcode`$=3\catcode`^=7}]
%% [-->;-------- T-Var \\
%%  (\rtag{:} \huhclause{?(\rtag{ck} $\Gamma$ (\rtag{ref} x))} \{\rtag{env-map} $\Gamma$ x\})]
%% \end{Verbatim}
%% \end{minipage}
%% \\
%% \\
%% {{\small\textsc{T-Abs}}\quad\quad{\LARGE ${ \frac{\Gamma, x\, : \,T_1 \, \vdash \, e \, : \, T_2}{(\lambda\,(x\,:\, T_1)\,e)\, : T_1 \rightarrow \,T_2}}$}} \quad& 
%% \begin{minipage}{2.5in}
%% \begin{Verbatim}[baselinestretch=.8,commandchars=\\\{\},codes={\catcode`$=3\catcode`^=7}]
%% [(\rtag{:} \bangclause{!(\rtag{ck} (\rtag{ext-env} $\Gamma$ x T1) e)} T2)
%%  -->;-------- T-Abs
%%  (\rtag{:} \huhclause{?(\rtag{ck} $\Gamma$ (\rtag{$\lambda$} x T1 e))} (\rtag{->} T1 T2))]
%% \end{Verbatim}
%% \end{minipage}
%% \\
%% \\
 
%% {{\small\textsc{T-App}}\quad\quad{\LARGE ${ \frac{{\Gamma \, \vdash \, e_0\,:\,T_0\, \rightarrow\, T_1}\quad e_1\,:\,T_0}{\Gamma \, \vdash \, (e_0~e_1)\,:\,T_1}}$ }}\quad& 
%% \begin{minipage}{2.5in}
%% \begin{Verbatim}[baselinestretch=.8,commandchars=\\\{\},codes={\catcode`$=3\catcode`^=7}]
%% [(\rtag{:} \bangclause{!(ck $\Gamma$ e0)} (\rtag{->} T0 T1))
%%  (\rtag{:} \bangclause{!(ck $\Gamma$ e1)} T0)
%%  -->;-------- T-App
%%  (\rtag{:} \huhclause{?(\rtag{ck} $\Gamma$ (\rtag{app} e0 e1))} T1)]
%% \end{Verbatim}
%% \end{minipage} \\
%% %\bottomrule
%% \end{tabular}
%% \caption{Simply-Typed Lambda Calculus (TAPL Fig. 9.1) and \slog{} Equivalent}
%% \label{fig:stlc}
%% \end{figure}


We begin with the simply-typed
$\lambda$-calculus~\cite{barendregt2013lambda,tapl}.  The syntax of
STLC terms and types is shown at the top of Figure~\ref{fig:stlc}. Our
presentation roughly follows Chapter~9 of Benjamin Pierce's
\textit{Types and Programming Languages}~\cite{tapl}; we use
Scheme-style syntax to more closely mirror the \slog{}
presentation. STLC extends the untyped $\lambda$-calculus:
$\lambda$-abstractions are annotated with types, and variable typing
defers to a typing environment which assigns types to type variables
and is subsequently extended at callsites.  STLC defines a notion of
``simple'' types including (always) arrow types between simple types
and (sometimes) base types (e.g., \textsf{nat}), depending on the
presentation.
 
The key creative challenge in translating a type system into \slog{}
lies in the operationalization of its control flow. For example,
consider the \textsc{T-Var} rule on the left side of
Figure~\ref{fig:stlc}: the rule is parametric over the typing
environment $\Gamma$, however a terminating analysis necessarily
inspects only a finite number of typing environments. The solution is
interpret the rules in a demand-driven way, using subfacts to defer
type checking until necessary. Operationally, each type rule is
predicated upon a message \texttt{\huhclause{?(\rtag{ck} $\Gamma$
    e)}}, which triggers type synthesis for \texttt{e} under the
typing environment $\Gamma$.  Using this technique, we may invoke the
analysis by including a fact \texttt{(\rtag{ck} $\Gamma$ e)}. Using
stratified negation, we may treat \texttt{(\rtag{:} (\rtag{ck}
  $\Gamma$ e) $\tau$)} as a decision procedure, like so:

\begin{Verbatim}[baselinestretch=1,commandchars=\\\{\},codes={\catcode`$=3\catcode`^=7}]
[(\rtag{success} $\Gamma$ e $\tau$) <-- (\rtag{typecheck} $\Gamma$ e $\tau$) (\rtag{:} \bangclause{!(\rtag{ck} $\Gamma$ e)} $\tau$)]
[(\rtag{failure} $\Gamma$ e $\tau$) <-- ~(\rtag{success} $\Gamma$ e $\tau$)]
\end{Verbatim}

Here, the inclusion of \texttt{(\rtag{typecheck} $\Gamma$ e $\tau$)}
forces the materialization of \texttt{\bangclause{!(\rtag{ck} $\Gamma$
    e)}}, and subsequently the enumeration of the type of each of its
subexpressions. Once the resulting type is materialized, we force its
unification with $\tau$ (implicitly using \slog{}'s intensional notion
of equality) and, when successful, generate \texttt{(\rtag{success}
  $\Gamma$ e $\tau$)}. The resulting \slog{} implementation
necessarily terminates because the analysis enumerates at most a
finite number of \texttt{(\rtag{ck} $\Gamma$ e)} facts (because of the
structural recursion on $e$ done by \rtag{:}), which in-turn forces
materialization of a finite number of \texttt{(\rtag{ext-env}
  $\Gamma$ x T)} facts, each of which forces a bounded
materialization of \rtag{env-map}.

\paragraph*{Natural Deduction and Per Martin-L\"of's Type Theory}

The well-known Curry-Howard isomorphism relates terms in pure
functional languages to proofs in an appropriate constructive
logic~\cite{Curry:1934}. For STLC, the Curry-Howard isomorphism tells
us that we may read our type checker as a decision procedure for
intuitionistic propositional logic. As an alternative (but equivalent)
perspective, we now consider how \slog{} may represent proofs in Per
Martin-L\"of's intuitionistic type theory (ITT)~\cite{Martin-Lof1996}.

By design, ITT cleanly separates propositions from their associated
derivations (proof objects). In \slog{}, we may represent derivations
as structured facts, obtaining the nested structure of derivations via
\slog{}'s subfacts. Adopting this perspective, checking a
natural-deduction proof in ITT involves propagating assumptions to
their usages (akin to the propagation achieved using maps in STLC).

\begin{figure}[h!]

\begin{tabular}{c|c}
{{\textsc{$\land$I}}\quad{\LARGE  ${ \frac{A \textit{ true} \quad \quad B \textit{ true}}{A \land B \textit{ true}}}$}} \quad& 
\begin{minipage}{2.5in}
{
\begin{Verbatim}[baselinestretch=.8,commandchars=\\\{\},codes={\catcode`$=3\catcode`^=7}]
[(\rtag{true} \bangclause{!(\rtag{ck} A-pf A)})
 (\rtag{true} \bangclause{!(\rtag{ck} B-pf B)})
 -->
 (\rtag{true} \huhclause{?(\rtag{ck} (\rtag{$\land$I} A-pf B-pf (\rtag{$\land$I} A B)) (\rtag{$\land$I} A B))})]
\end{Verbatim}
}
\end{minipage}
\end{tabular}
\caption{The introduction rule for $\land$ in ITT.}
\label{fig:itt}
\end{figure}

Figure~\ref{fig:itt} shows the introduction form for $\land$
(left) and its corresponding \slog{} transliteration (right). Checking
a derivation of $\land$I forces the checking of each sub-derivation,
and (upon success) populates the \rtag{true} relation with the
appropriate derivation.

Implication in ITT is managed by introducing, and then discharging,
assumptions: $A \supset B$ holds whenever $B$ may be proven by
assuming $A$. Figure~\ref{fig:impl} details the implication rule in
ITT (left) and \slog{} (right): the introduced hypothesis (named $u$
in $\supset{}\!I$) is subsequently discharged to produce an
assumption-free proof of $A \supset B~\textit{true}$. In \slog{}, the
rule for $\supset{}\!I$ introduces an assumption by forcing the
materialization of \texttt{(\rtag{assuming} A pf-B)}---other rules
then ``push down'' the assumption $A~\textit{true}$ to their eventual
uses (top right of Figure~\ref{fig:impl}), performing transitive
closure of assumptions to their usages in an on-demand fashion.

\begin{figure}[h!]
\begin{tabular}{c|c}
%% -- u
%% A  
%% ..
%% B 
%% -- 
%% A -> B

${\supset{}\!I}\quad \frac { \frac{}{\begin{array}{c}A~\textit{true}\\\vdots\\ B~\textit{true}\end{array}}~u } {\begin{array}{c}A \supset B~\textit{true}\end{array}}$

&
\begin{minipage}{2.5in}
{
\vspace{-.5in}
\begin{Verbatim}[baselinestretch=.8,commandchars=\\\{\},codes={\catcode`$=3\catcode`^=7}]
;; Propagate assumptions
(\rtag{true} \huhclause{?(\rtag{ck} (\rtag{assuming} P (\rtag{assumption} P)) P)} P)
...

[(\rtag{true} \bangclause{!(\rtag{ck} (\rtag{assuming} A pf-B) B)})
 -->
 (\rtag{true} \huhclause{?(\rtag{ck} (\rtag{$\supset$I} (\rtag{$\supset$} A B) pf-B) (\rtag{$\supset$} A B))})]
\end{Verbatim}
}
\end{minipage}
\end{tabular}
\caption{Implication in ITT and \slog{}}
\label{fig:impl}
\end{figure}


\paragraph*{First-Order Dependent Types: \lf{}}

\begin{figure}
\begin{displaymath}
\begin{tabular}{lrcllrcl}
\textit{\lf{} Contexts}& $\Gamma$ & $::=$ & $\varnothing$ &\textit{\lf{} Types} & $\tau \in T$ & $::=$& $X$ \\
&     &  $\mid$ & $\Gamma, x \! : \! T$ & && $\mid$& $\Pi x \! : \! T.~ T$ \\
&     &  $\mid$ & $\Gamma, x \! ::\!  K$ & &&$\mid$& $(T~ e)$ \\
\textit{\lf{} Kinds} & $K$ & $::=$ & $\Pi x \! : \! T. K \mid \ast$ \\ 
\end{tabular}
\end{displaymath}
\begin{tabular}{c|c}
%\toprule \\
{{\small\textsc{TA-Abs}}\quad\quad{\LARGE ${ \frac{\Gamma\, \vdash\, S\, ::\, \ast \quad \Gamma, \,x\, :\, S\, \vdash\, t\, : \,T}{\Gamma\, \vdash \, (\lambda (x\,:\,S) t) \,:\, \Pi x\, :\, S. T}}$}} \quad& 
\begin{minipage}{2.5in}
{\footnotesize
\begin{Verbatim}[baselinestretch=.8,commandchars=\\\{\},codes={\catcode`$=3\catcode`^=7}]
[(\rtag{::} \bangclause{!(\rtag{ck-k} $\Gamma$ S)} (\rtag{$\ast$}))
 (\rtag{:} \bangclause{!(\rtag{ck-t} (ext-env $\Gamma$ x S) t)} T)
 -->;------
 (\rtag{:} \bangclause{?(\rtag{ck-t} $\Gamma$ (\rtag{$\lambda$} x S t))} (\rtag{$\Pi$} x S T))]
\end{Verbatim}
}\end{minipage}
\\
\\
  {{\small\textsc{TA-App}}\quad\quad{\LARGE ${ \frac{\Gamma \, \vdash \, t_1 \,:\, \Pi x : S_1 . \, T \quad \Gamma \, \vdash \, t_2\, : \,S_2 \quad \Gamma\, \vdash\, S_1\, \equiv\, S_2}{\Gamma\, \vdash\, (t_1~t_2)\, : \,[ x \,\mapsto\, t_2 ] \, T}}$}}\quad&
\begin{minipage}{2.5in}
{\footnotesize
\begin{Verbatim}[baselinestretch=.8,commandchars=\\\{\},codes={\catcode`$=3\catcode`^=7}]
[(\rtag{:} \bangclause{!(\rtag{ck-t} $\Gamma$ t1)} (\rtag{$\Pi$} x S1 T))
 (\rtag{:} \bangclause{!(\rtag{ck-t} $\Gamma$ t2)} S2)
 (\rtag{true} \bangclause{!(\rtag{===} $\Gamma$ S1 S2)})
 -->;------
 (\rtag{:} \huhclause{?(\rtag{ck-t} $\Gamma$ (\rtag{app} t1 t2))}
   \{\rtag{subst} \bangclause{!(\rtag{do-subst} T x t2)}\})]
\end{Verbatim}
}\end{minipage} \\
\\
  {{\small\textsc{KA-App}}\quad\quad{\LARGE ${ \frac{\Gamma \, \vdash \, S \,::\, \Pi x : T_1 . \, K \quad \Gamma \, \vdash \, t\, : \,T_2 \quad \Gamma\, \vdash\, T_1\, \equiv\, T_2}{\Gamma\, \vdash\, (S~t)\, : \,[ x \,\mapsto\, t ] \, K}}$}}\quad&
\begin{minipage}{2.5in}
{\footnotesize
\begin{Verbatim}[baselinestretch=.8,commandchars=\\\{\},codes={\catcode`$=3\catcode`^=7}]
[(\rtag{::} \bangclause{!(\rtag{ch-k} $\Gamma$ S)} (\rtag{$\Pi$} x T1 K))
 (\rtag{:} \bangclause{!(\rtag{ck-t} $\Gamma$ t)} T2)
 (\rtag{true} \bangclause{!(\rtag{===} $\Gamma$ T1 T2)})
 -->;------
 (\rtag{:} \huhclause{?(\rtag{ch-t} $\Gamma$ (\rtag{type-app} S t))}
   \{subst !(do-subst K x t)\})]
\end{Verbatim}
}\end{minipage} \\
\\
%\bottomrule
\begin{minipage}{2.75in}
{\small
\begin{Verbatim}[baselinestretch=.8,commandchars=\\\{\},codes={\catcode`$=3\catcode`^=7}]
[(\rtag{->wh} \bangclause{!(\rtag{do->wh} t1)} t1')
 -->;------
 (\rtag{->wh} \huhclause{?(\rtag{do->wh} (\rtag{app} t1 t2))} (\rtag{app} t1' t2))]
\end{Verbatim}
}
\end{minipage}
&
\quad
\quad
\begin{minipage}{2.75in}
{\small
\begin{Verbatim}[baselinestretch=.8,commandchars=\\\{\},codes={\catcode`$=3\catcode`^=7}]
(\rtag{->wh} \huhclause{?(\rtag{do->wh} (\rtag{app} ($\lambda$ x T1 t1) t2))}
  \{subst \bangclause{!(do-subst t1 x t2)}\})
\end{Verbatim}
}\end{minipage} 
\end{tabular}

\caption{\lf{}: Syntax (top), selected rules (left) and \slog{} (right).}
\label{fig:lf}
\end{figure}

The Edinburgh Logical Framework (\lf{}) is a dependently-typed
$\lambda$-calculus~\cite{Harper:93}. It is a first-order dependent
type system, in the sense that it stratifies its objects into kinds, types
(families), and terms (values)---kinds may quantify over types, but
% Todo: Why put this idea here right up front?
not over other kinds. The syntax of \lf{} is detailed at the bottom of
Figure~\ref{fig:lf}---it extends STLC with kinds, which are either
$\ast$ or (type families) $\Pi x : T. K$, where $T$ is a simple type
(of kind $\ast$). \lf{} generalizes the arrow type to the dependent
product: $\Pi x: T. T$. System \lf{} enjoys several decidability
properties which make it particularly amenable to implementation in
\slog{}. The first is strong normalization, which implies that
reduction sequences for well-typed terms in our implementation will be
finite. The second is \lf{}'s focus on canonical forms and hereditary
substitution~\cite{harper:2007}. In \lf{}, terms are canonicalized to
weak-head normal form (WHNF); this choice enables inductive reasoning
on these canonical forms, and this methodology forms the basis for
% Todo: Say a bit more about Twelf? It's meant for theorem proving or programming?
Twelf~\cite{pfenning1998twelf}.

The \textit{judgments-as-types} principle interprets type checking for
\lf{} as proving theorems in intuitionstic predicate logic; Using this
principle, we may define traditional constructive connectives (such as
$\land$, $\lor$, and $\exists$) via type families and their associated
rules. For example, including in $\Gamma$ a binding $\land \mapsto
\Pi\, P : \textit{prop}.~ \Pi\, Q : \textit{prop}.~ \ast$ allows using
the constructor $\land$, though $\land$ must be instantiated with a
suitable $P$ and $Q$, which must necessarily be of some sort (e.g.,
\textit{prop}) also bound in $\Gamma$.

We have performed a transliteration of \lf{} as formalized in Chapter
2 of \textit{Advanced Topics in Types and Programming Languages
  (ATAPL)}~\cite{atapl}. Our transliteration (from pages 57--58)
consists of roughly 150 lines of \slog{}
code. Figure~\ref{fig:lf} details several of the key
rules. \textsc{TA-Abs} introduces a $\Pi$ type, generalizing the
\textsc{T-Abs} rule in Figure~\ref{fig:stlc}. The \textsc{TA-App}
applies a term $t_1$, of a dependent product type $\Pi x :S_1 ~.~T$,
whenever the input $t_2$ shares an equivalent type, $S_2$. The notion
of equality here is worth mentioning: $\equiv$ demands reduction of
its arguments to WHNF---\lf{} is constructed to identify terms under
WHNF, thus ensuring $\equiv$ will terminate as long as the term is
typeable. Reduction to WHNF is readily implemented in \slog{}; two
exemplary rules detailed at the bottom of Figure~\ref{fig:lf}
% Todo: I changed this from fig:lf-example; correct?
outline the key invariant in WHNF: reduce down the leftmost spine,
eliminating $\beta$-redexes via application. Equality checks are
demanded by the \textsc{TA-App} and \textsc{KA-App} rules, and force
normalization of their arguments to WHNF before comparison of
canonical forms, generating a witness in \rtag{true} before triggering
the head of the rule.


%% \begin{tabular}{c|c|c}
%% \begin{align}
%% \forall \, Q,&\, \forall\, P.& \\
%%   &\,Q \land P \rightarrow P \land Q
%% \end{align}
%% &
%% \begin{minipage}{2.75in}
%% {\small
%% \begin{Verbatim}[baselinestretch=.6,commandchars=\\\{\},codes={\catcode`$=3\catcode`^=7}]
%% (tcheck
%%  (extend
%%   (extend (bot) (var (prop)) (star))
%%   (var (land)) (PiK (x) (var (prop))
%%                     (PiK (y) (var (prop)) (star))))
%%  (lam (x) (var (prop))
%%       (lam (y) (var (prop))
%%            (type-app (var (land)) (ref (x))))))
%% \end{Verbatim}
%% }
%% \end{minipage}
%% &
%% \quad
%% \quad
%% \begin{minipage}{2.75in}
%% {\small
%% \begin{Verbatim}[baselinestretch=.6,commandchars=\\\{\},codes={\catcode`$=3\catcode`^=7}]
%% (tcheck
%%  (extend
%%   (extend (bot) (var (prop)) (star))
%%   (var (land)) (PiK (x) (var (prop))
%%                     (PiK (y) (var (prop)) (star))))
%%  (lam (x) (var (prop))
%%       (lam (y) (var (prop))
%%            (type-app (var (land)) (ref (x))))))
%% \end{Verbatim}
%% }\end{minipage} 
%% \end{tabular}

